%% BioMed_Central_Tex_Template_v1.06
%%                                      %
%  bmc_article.tex            ver: 1.06 %
%                                       %

%%IMPORTANT: do not delete the first line of this template
%%It must be present to enable the BMC Submission system to
%%recognise this template!!

%%%%%%%%%%%%%%%%%%%%%%%%%%%%%%%%%%%%%%%%%
%%                                     %%
%%  LaTeX template for BioMed Central  %%
%%     journal article submissions     %%
%%                                     %%
%%          <8 June 2012>              %%
%%                                     %%
%%                                     %%
%%%%%%%%%%%%%%%%%%%%%%%%%%%%%%%%%%%%%%%%%


%%%%%%%%%%%%%%%%%%%%%%%%%%%%%%%%%%%%%%%%%%%%%%%%%%%%%%%%%%%%%%%%%%%%%
%%                                                                 %%
%% For instructions on how to fill out this Tex template           %%
%% document please refer to Readme.html and the instructions for   %%
%% authors page on the biomed central website                      %%
%% http://www.biomedcentral.com/info/authors/                      %%
%%                                                                 %%
%% Please do not use \input{...} to include other tex files.       %%
%% Submit your LaTeX manuscript as one .tex document.              %%
%%                                                                 %%
%% All additional figures and files should be attached             %%
%% separately and not embedded in the \TeX\ document itself.       %%
%%                                                                 %%
%% BioMed Central currently use the MikTex distribution of         %%
%% TeX for Windows) of TeX and LaTeX.  This is available from      %%
%% http://www.miktex.org                                           %%
%%                                                                 %%
%%%%%%%%%%%%%%%%%%%%%%%%%%%%%%%%%%%%%%%%%%%%%%%%%%%%%%%%%%%%%%%%%%%%%

%%% additional documentclass options:
%  [doublespacing]
%  [linenumbers]   - put the line numbers on margins

%%% loading packages, author definitions

%\documentclass[twocolumn]{bmcart}% uncomment this for twocolumn layout and comment line below
\documentclass{bmcart}

%%% Load packages
%\usepackage{amsthm,amsmath}
%\RequirePackage{natbib}
%\RequirePackage[authoryear]{natbib}% uncomment this for author-year bibliography
%\RequirePackage{hyperref}
\usepackage[utf8]{inputenc} %unicode support
%\usepackage[applemac]{inputenc} %applemac support if unicode package fails
%\usepackage[latin1]{inputenc} %UNIX support if unicode package fails

% Packages
\usepackage{latexsym}
\usepackage{epic,eepic}
\usepackage{times}
\usepackage{amsmath}
\usepackage{amsfonts}
\usepackage{multirow}
\usepackage{graphicx}
\usepackage{algorithm}
\usepackage{algpseudocode}
\usepackage{xcolor}
\usepackage{ulem}
\usepackage{amssymb}

% Tunning
%------------------
\usepackage[12pt]{moresize}
\let\labelindent\relax
\usepackage{enumitem}
\algtext*{EndIf}% Remove "end if" text
\algtext*{EndFor}% Remove "end if" text
%\setlength{\belowcaptionskip}{10.0pt}
%\setlength{\abovecaptionskip}{10.0pt}
%\setlength{\textfloatsep}{10.0pt}
%\setlength{\floatsep}{10.0pt}
% TUNNING
%\setstretch{0.94}
%\fontsize{3.85mm}{3.85mm}\selectfont


\newcommand{\pf}[1]{\langle#1\rangle}
\newcommand{\noproof}{\hfill $qed$}
\newcommand{\ignore}[1]{}

\definecolor{gray50}{gray}{0.15}
\definecolor{verylightgray}{gray}{0.90}


\newtheorem{definition}{Definition} % [section]
\newtheorem{example}{Example} % [section]
\newtheorem{Lemma}{Lemma} % [section]

\newcommand{\pivot}[1]{\mathbin{\, {#1} \,}}
\newcommand{\Pivot}[1]{\mathbin{\; {#1} \;}}
\let\from=\leftarrow

% Pretty ref
\usepackage{prettyref}
\newrefformat{def}{Definition~\ref{#1}}
\newrefformat{fig}{Figure~\ref{#1}}
\newrefformat{tab}{Table~\ref{#1}}
%%\newrefformat{pro}{Property~\ref{#1}}
%%\newrefformat{pps}{Proposition~\ref{#1}}
%\newrefformat{lem}{Lemma~\ref{#1}}
%\newrefformat{th}{Theorem~\ref{#1}}
%\newrefformat{sec}{Section~\ref{#1}}
%%\newrefformat{subsec}{Subsect.~\ref{#1}}
%%\newrefformat{suppl}{Appendix~\ref{#1}}
\newrefformat{ex}{Example~\ref{#1}}
%%\newrefformat{eq}{Eq.~\eqref{#1}}
\def\pref{\prettyref}

%Algorithme
\usepackage{algorithm}
\usepackage{algpseudocode}


%Package CAV
\usepackage[english]{babel}
\usepackage{stmaryrd} % Maths (crochets doubles)
\usepackage{url}     % Mise en forme + liens pour URLs
%\usepackage{array}   % Tableaux évolués
\usepackage{setspace}

% Commande perso
\newcommand{\ie}{i.e.,\ }
\newcommand{\eg}{e.g.,\ }
\newcommand{\resp}{resp.\ }
\newcommand{\nbld}{\nobreakdash--}
\newcommand{\refl}[1]{line~\ref{#1}}
\newcommand{\refll}[2]{lines~\ref{#1}\nobreakdash--\ref{#2}}
\newcommand{\Refl}[1]{Line~\ref{#1}}
\newcommand{\Refll}[2]{Lines~\ref{#1}\nobreakdash--\ref{#2}}



% Citation
\usepackage{cite}

%marge
\newenvironment{changemargin}[2]{\begin{list}{}{%
\setlength{\topsep}{0pt}%
\setlength{\leftmargin}{0pt}%
\setlength{\rightmargin}{0pt}%
\setlength{\listparindent}{\parindent}%
\setlength{\itemindent}{\parindent}%
\setlength{\parsep}{0pt plus 1pt}%
\addtolength{\leftmargin}{#1}%
\addtolength{\rightmargin}{#2}%
}\item }{\end{list}}

% Table
\usepackage{colortbl}
\definecolor{gray50}{gray}{0.15}
\definecolor{verylightgray}{gray}{0.90}

%\newtheorem{definition}{Definition} % [section]
%\newtheorem{example}{Example} % [section]
%\newcommand{\pivot}[1]{\mathbin{\, {#1} \,}}
%\newcommand{\Pivot}[1]{\mathbin{\; {#1} \;}}
%\let\from=\leftarrow

% **** ASP language ***
\usepackage{listings}
\lstdefinelanguage{ASP}{\^^M}
}
% Définition des styles de tous les listings du document
\lstset{
language=ASP,
basicstyle=\small\ttfamily,
columns=fullflexible,
keywordstyle=\bfseries,
firstnumber=last,
keepspaces=true,
numbersep=2pt,
numberstyle=\tiny\color{darkgray},
}
\renewcommand{\thelstnumber}{\the\value{lstnumber}}
%% fin définition

% Styles ASP inline
\newcommand{\ASPnot}{\textbf{not}~}



%%%%%%%%%%%%%%%%%%%%%%%%%%%%%%%%%%%%%%%%%%%%%%%%%
%%                                             %%
%%  If you wish to display your graphics for   %%
%%  your own use using includegraphic or       %%
%%  includegraphics, then comment out the      %%
%%  following two lines of code.               %%
%%  NB: These line *must* be included when     %%
%%  submitting to BMC.                         %%
%%  All figure files must be submitted as      %%
%%  separate graphics through the BMC          %%
%%  submission process, not included in the    %%
%%  submitted article.                         %%
%%                                             %%
%%%%%%%%%%%%%%%%%%%%%%%%%%%%%%%%%%%%%%%%%%%%%%%%%


\def\includegraphic{}
\def\includegraphics{}

%%% A ENLEVER pour les modifications en Rouge
\usepackage{color}
\newcommand{\Emna}[1]{\textcolor{red}{#1}}
\newcommand{\EmnaRq}[1]{\textcolor{green}{#1}}
\newcommand{\Maxime}[1]{\textcolor{violet}{#1}}

%%% Put your definitions there:
\startlocaldefs
\endlocaldefs
%% ------------------------------------------------------------------------------- %%

% les inputs files
\input{macros/macros}
\input{macros/macros-ph}
\input{macros/tikzstyles2.tex}

%%% Begin ...
\begin{document}

%%% Start of article front matter
\begin{frontmatter}

\begin{fmbox}
\dochead{Research}

%%%%%%%%%%%%%%%%%%%%%%%%%%%%%%%%%%%%%%%%%%%%%%
%%                                          %%
%% Enter the title of your article here     %%
%%                                          %%
%%%%%%%%%%%%%%%%%%%%%%%%%%%%%%%%%%%%%%%%%%%%%%

\title{ASP-based method for enumerating attractors in asynchronous multi-valued networks }

%%%%%%%%%%%%%%%%%%%%%%%%%%%%%%%%%%%%%%%%%%%%%%
%%                                          %%
%% Enter the authors here                   %%
%%                                          %%
%% Specify information, if available,       %%
%% in the form:                             %%
%%   <key>={<id1>,<id2>}                    %%
%%   <key>=                                 %%
%% Comment or delete the keys which are     %%
%% not used. Repeat \author command as much %%
%% as required.                             %%
%%                                          %%
%%%%%%%%%%%%%%%%%%%%%%%%%%%%%%%%%%%%%%%%%%%%%%

\author[
   addressref={univ1},                   % id's of addresses, e.g. {aff1,aff2}
 %  corref={aff1},                       % id of corresponding address, if any
 %  noteref={n1},                        % id's of article notes, if any
   email={emna.ben-abdallah@irccyn.ec-nantes.fr}   % email address
]{\inits{EB}\fnm{Emna} \snm{Ben Abdallah}}
\author[
   addressref={univ2},
   email={maxime.folschette@unice.fr}
   ]{\inits{MF}\fnm{Maxime} \snm{Folschette}}
\author[
   addressref={univ1},
   email={olivier.roux@irccyn.ec-nantes.fr}
]{\inits{OR}\fnm{Olivier} \snm{Roux}}
\author[
   addressref={univ1, univ3},
   email={morgan.magnin@irccyn.ec-nantes.fr}
]{\inits{MM}\fnm{Morgan} \snm{Magnin}}

%%%%%%%%%%%%%%%%%%%%%%%%%%%%%%%%%%%%%%%%%%%%%%
%%                                          %%
%% Enter the authors' addresses here        %%
%%                                          %%
%% Repeat \address commands as much as      %%
%% required.                                %%
%%                                          %%
%%%%%%%%%%%%%%%%%%%%%%%%%%%%%%%%%%%%%%%%%%%%%%

\address[id=univ1]{%                           % unique id
  \orgname{LUNAM Universit\'e, \'Ecole Centrale de Nantes, IRCCyN UMR CNRS 659
(Institut de Recherche en Communications et Cybern\'etique de Nantes)}, % university, etc
  \street{1 rue de la No\"e},                     %
  \postcode{44321}                                % post or zip code
  \city{Nantes},                              % city
  \cny{France}                                    % country
}
\address[id=univ2]{%
  \orgname{I3S, UMR 7271 --- Universit\'e de Nice-Sophia Antipolis},
 % \street{D\"{u}sternbrooker Weg 20},
 % \postcode{24105}
  \city{Nice},
  \cny{France}
}
\address[id=univ3]{%                           % unique id
  \orgname{National Institute of Informatics}, % university, etc
  \street{2-1-2, Hitotsubashi, Chiyoda-ku},                     %
  \postcode{101-8430}                                % post or zip code
  \city{Tokyo},                              % city
  \cny{Japan}                                    % country
}


\begin{abstractbox}

\begin{abstract} % abstract
\parttitle{Background} 
This paper addresses the problem of finding cycles and attractors in asynchronous multi-valued automata networks which are used for the modeling of biological regulatory networks. These networks are models used to study the interactions between different genes in genetic regulatory networks. An attractor is a terminal component in the state transition graph which can be either a steady state (singleton) or a composition of cycles (non-singleton). Studying the effect of a disease or a mutation on an organism requires to find the attractors in the model to understand the long-lasting behaviors.
%This motivates research on algorithms for finding attractors.
\parttitle{Results} 
We present a computational Answer Set Programming (ASP) based method to identify all attractors in fully asynchronous updating mode and without any network reduction. The method goes through a complete enumeration of the states forming each attractor without the necessity to simulate the whole network or construct the state transition graph. We performed extensive computational experiments which showed good performance and fitted the expected theoretical results.
\parttitle{Conclusion} 
The originality of our approach relies in the exhaustive enumeration of all possible states verifying the attractor properties thanks to the use of ASP. The merits of our methods are illustrated by applying them to biological examples of various sizes and comparing the results with some existing approaches. It turns out that our approach succeeds in processing large models, that is, up to tens of components and interactions.
\end{abstract}

\begin{keyword}
\kwd{Biological Regulatory Network}
\kwd{multiple-valued networks}
\kwd{attractor}
\kwd{steady state}
\kwd{cycle}
\kwd{Answer Set Programming}
\end{keyword}


\end{abstractbox}
%
\end{fmbox}% uncomment this for twcolumn layout

\end{frontmatter}



\section{Introduction}
\Emna{A completer avec les bonne références}
\Maxime{Ajouter quelques généralités sur les BRN}

\Maxime{Je ne suis pas d'accord sur la notion de “cyclic attractor”: je crois que tu confonds “trap domain” (un ensemble d'états dont on ne peut pas s'échapper ; un attracteur est un trap domain minimal) et “cyclic attractor” (qui est un attracteur, donc minimal, composé d'un seul cycle).}

In recent decades, rapidly evolving new technologies have been producing a massive amount of biological data (genomics, proteomics...). This lead to dramatic developments in systems biology which could sustain on this data. Indeed, from the perspective of understanding the nature of a cellular function, it is essential to study the behavior of its components (genes, DNAs, proteins...) with a global view rather than an individual one, and take into account the fact that activities and expressions of cellular components are not isolated but dependent on each other.

In this focus, the long-term behavior of regulatory networks dynamics is of particular interest~\cite{wuensche1998genomic}.
Indeed, at any moment, a system lies into a certain \emph{basin of attraction}, which is an area of its dynamics that cannot be escaped.
When evolving, the system eventually falls into a new and smaller basin of attraction, thus reducing its of possible future behaviors as previous ones become unreachable.
This phenomenon may depend on biological switches or other complex phenomena
and it explains why, for any initial condition and on the long run, any system will eventually evolve into a final situation in which the number of possible behaviors is the most restricted. \Emna{ref?}
The system has then reached a minimal basin of attraction.
Such constructs have been interpreted as distinct responses of the organism, such as distinct cell types in multicellular organisms. \Maxime{Ref ? Peut-être détailler un peu plus pour insister sur l'intérêt des attracteurs ?}

Various kinds of mathematical models have been proposed for the modeling of Biological Regulatory Networks (BRNs). These models include neural networks, differential equations, Petri nets, Boolean networks (as proposed by Kauffman \cite{stuart1993origins}), probabilistic Boolean networks, and other multi-valued models such Synchronous Automata Networks. In this paper, we focus on Process Hitting \Maxime{ref}, an asynchronous formalism which is a restriction of Automata Networks, and that especially encompasses the framework of René Thomas \Emna{ref}. Indeed, qualitative frameworks have received substantial attention, because of their capacity to capture the switching behavior of genetic or biological processes, and therefore their long-term behavior. This is why we use a qualitative representation for the study of basins of attraction.
More particularly, in a qualitative framework, a minimal basin of attraction can take two different forms: it can be either a \emph{stable state} (a state from which the system doesn't evolve anymore, also called a fixed point) or an \emph{attractor} (a minimal set of states that loops indefinitely and cannot be escaped).

%%%%%%%%%

The computational problem of finding all attractors in a BRN is difficult. Even the simpler problem of deciding whether the system has a fixed point (which can be seen as the smallest possible kind of attractor) is NP-hard.
Based on the former \Maxime{Il manque une ref ?}, many studies have proven that the computational way of attractors in BRNs is a NP-hard problem \cite{klemm2005stable,akutsu2012finding}. \Maxime{ref ?}

Therefore, it is meaningful to identify attractors. Of course, these can be done by examining all possible states of a network with an exhaustive approach. For instance, finding an attractor or a stable state can be performed by randomly selecting an initial state and following a long enough trajectory.
However, listing all attractors with this method can be too time consuming even for small networks: $2^n$ initial states have to be considered for a Boolean networks of size $n$; and multi-valued networks raise this complexity even more. Furthermore, enough computation have to be performed to ensure that all trajectories have been explored and all attractors were found.
%%%%%%%

\Emna{Les travaux qui ont été faits concernant la recherche des attracteurs}

Our goal in this paper is to develop exhaustive methods to analyze a Biological Regulatory Network (BRN) modeled in Process Hitting (PH). With respect to PH dynamics, we exhibit three kinds of results:
\begin{itemize}
\item[-] Finding all possible steady states of a BRN,
\item[-] Computing all cycles with fixed size $N \in \mathbb{N}$,
\item[-] Enumerating all attractors of size $N \in \mathbb{N}$.
\end{itemize}
The particularity of our contribution relies in the use of Answer Set Programming
(ASP) \cite{baral2003knowledge}
to compute the results.
This declarative programming framework has proved efficient
to tackle models with a large number of components and parameters.
Our aim here is to assess its potential w.r.t.\ the computation
of some dynamical properties of PH models.
We chose the PH framework because it allows to represent a wide range of dynamical multi-valued models in the asynchronous framework, and the particular form of its actions
can be easily represented using ASP,
with exactly one fact per action.

This paper is organized as follows.
\pref{sec:defs} presents the main definitions related to the Process Hitting and the particular constructs that we will seek: fixed points and attractors,
and \pref{sec:asp} briefly present the Answer Set Programming.
\pref{sec:fixpoint} details our method that allows to translate a PG into ASP and find all the fixed points in such a model
and \pref{sec:attractors} explains how to enumerate all cycles in a model and use this result to find all the attractors.
\pref{sec:results} sums up the results obtained with our methods on several models of different sizes.
Finally, \pref{sec:ccl} concludes and gives some perspectives to this work.


\section{Preliminary definitions}
\label{sec:defs}
\subsection{Process Hitting}
\pref{def:PH} introduces the Process Hitting~(PH)~\cite{PMR10-TCSB}
which allows to model a finite number of local levels,
called \emph{processes},
grouped into a finite set of components, called \emph{sorts}.
A process is noted $a_i$, where $a$ is the sort's name,
and $i$ is the process identifier within sort $a$.
At any time, exactly one process of each sort is \emph{active},
and the set of active processes is called a \emph{state}.

The concurrent interactions between processes are defined by a set of \emph{actions}.
Each action is responsible for the replacement of one process by another of the same sort
conditioned by the presence of a set of other process, at least one, in the current state.
An action is denoted by $\PHfrappe{A}{b_j}{b_k}$, which is read as "all the processes in $A$ cooperate to \emph{hit} $b_j$ and make it \emph{bounce} to $b_k$'',
where $b_j$, $b_k$ are processes of sorts $b$.
and we call the processes of $A$, $b_j$ and $b_k$  respectively \emph{hitters}, \emph{target} and
\emph{bounce} of the action.
We also call a \emph{self-hit} any action with one hitter such that the hitter and target sorts are the same,
that is, of the form: $\PHfrappe{b_j}{b_j}{b_k}$.

The PH is therefore a restriction of synchronous automata networks with each transition
changing the local state of exactly one automaton,
therefore being an asynchronous restriction.
This restriction on the actions was originally chosen to permit
the development of efficient static analysis methods
based on abstract interpretation~\cite{PMR12-MSCS}.

\begin{definition}[Process Hitting]\label{def:PH}
  A \emph{Process Hitting} is a triple $(\PHs,\PHl,\PHa)$ where:
  \begin{itemize}
    \item  $\PHs = \{a,b,\dots\}$ is the finite set of \emph{sorts};
    \item  $\PHl = \prod_{a\in\PHs} \PHl_a$ is the set of \emph{states} where
      $\PHl_a = \{a_0,\dots,a_{l_a}\}$
      is the finite set of \emph{processes} of sort $a\in\Sigma$
      and $l_a$ is a positive integer, with $a\neq b\Rightarrow \PHl_a \cap \PHl_b = \emptyset$;
    \item $\PHa$ = \{ $\PHfrappe{A}{b_j}{b_k}$ with $A \in \PHl^{\diamond} \wedge b \in \PHs \wedge b_j \neq b_k \wedge$ if $b_j \in A \Rightarrow A=b_j$\} is the finite set of \emph{actions},
    with $\PHl^{\diamond}$ the set of all the sub-states of $\PHl$
    (that is, containing at most one process from each sort).
      %$\PHa$ = \{ $\PHfrappe{A}{b_j}{b_k}$ with $A \in \PHl^{\diamond} \wedge b \in \PHs \wedge b_j \neq b_k \wedge$ if $b_j \in A \Rightarrow A=b_j$ \}.With $\PHl^{\diamond}$ the set of all the sub-states of $\PHl$. 
  \end{itemize}
\end{definition}

\begin{example}
\pref{fig:ph} represents a PH model with five sorts: $a$, $b$, $c$, $d$ and $e$.
\begin{figure}[!h]
  \centering
  \scalebox{1.3}{
  \begin{tikzpicture}[apdotsimple/.style={apdot}]

   \TSort{(3.5,0.5)}{a}{2}{r}
    \TSort{(3,4.5)}{b}{3}{t}
    \TSort{(7,1)}{e}{2}{r}  
    \TSort{(0,3)}{c}{2}{r}
    \TSort{(0.5,0)}{d}{2}{r}
    
    \THit{d_1}{selfhit}{d_1}{.north west}{d_0}
       
   % \THit{e_1}{selfhit}{e_1}{.north east}{e_2}
% b en 3 niveaux avec un selfhit de 1 à 2
   \THit{d_0}{}{a_0}{.west}{a_1}
   \TActionPlur{a_1, b_1}{e_0.north west}{e_1.west}{}{5.5,2.5}{left}
   \TActionPlur{a_1, c_1}{b_0.south}{b_1.south west}{}{2,3}{right}

    \path[bounce, bend right]
      % \TBounce{d_1}{}{d_0}{.north}
   \TBounce{d_1}{}{d_0}{.north west}
  %    \TBounce{e_1}{}{e_2}{.north east}
    ;
   \path[bounce, bend left]
   \TBounce{a_0}{}{a_1}{.south west}
%	\TBounce{z_0}{bend left=90}{z_1}{.south east}
    ;
    \TState{a_0, b_0, c_1, d_1, e_0}
  \end{tikzpicture}
  }
  \caption{\label{fig:ph}
An example of PH model with five sorts: $a$, $b$, $c$, $d$ and $e$. Boxes represent the sorts, the biological components, circles represent the processes (component levels), and the actions that model the dynamic behavior are depicted by pairs of arrows in solid and dotted lines. $a$, $d$, $c$ and $e$ are all either at level $0$ or $1$, and the sort $b$ has $3$ levels $0$, $1$, $2$. The grayed processes stand for the possible initial state $\PHstate{a_0, b_0, c_1, d_1, e_0}$.
  }
\end{figure}

\end{example}

A \emph{state} of the network is a set of active processes containing a single process of each sort.
The active process of a given sort $a \in \PHs$ in a state $\PHst \in \PHl$
is noted $\PHget{\PHst}{a}$.
For any given process $a_i$ we also note: $a_i \in \PHst$ if and only if $\PHget{\PHst}{a} = a_i$. It means that the biological component $a$ is in the condition labeled $i$ within state $\PHst$.

\subsection{Attractors in Process Hitting}

The study of the dynamics of biological networks was the focus of many works, explaining the diversity of network modelings and the different methods developed in order to identify attractors \cite{skodawessely2011finding, zhang2007algorithms, mushthofa2014asp, akutsu2012finding, berntenis2013detection}.
In this paper we focus on several sub-problems related to such this in automata networks: the steady states, the cycles and the attractors. %Then we verify the reachability of these attractors. \\
In the following, we consider an automata network modeled in PH $\PH=(\PHs,\PHl,\PHa)$,
and we formally define these properties.
%How to verify them with the help of ASP is the subject of the rest of this paper.
%and explain how they could be verified on a such network.

\begin{definition} [Playable action]
\label{def:playableAction}
Let $\PH = (\PHs,\PHl,\PHa)$ be a Process Hitting and $\PHst \in \PHl$ a state of $\PH$.
We say that the action $h = \PHfrappe{a_i}{b_j}{b_k} \in \PHa$
is \emph{playable in state $\PHst$} if and only if
$A \subseteq \PHst$ and $b_j \in \PHst$ (\ie $\forall a_i \in A$, $\PHget{\PHst}{a} = a_i$ and $\PHget{\PHst}{b}=b_j$).
\end{definition}

The set of playable action in a given PH state give the set of transitions starting from this state in the state-transition graph.
Indeed, in each state, several action may be playable, but only one is effectively played and is responsible for the evolution of the system from one state to another.

\begin{definition}
The resulting state after playing an action $h= \PHfrappe{A}{b_j}{b_k}$ in a state $\PHst$
is called a \emph{successor} of $\PHst$ and
is denoted by $(\PHst \play h)$,
where $\PHget{(\PHst \play h)}{b} = b_k$ and
$\forall c \in \PHs, c \neq b \Rightarrow \PHget{(\PHst \play h)}{c}=\PHget{\PHst}{c}$.
\end{definition}

Since we are considering only the \emph{asynchronous} updating scheme (\Emna{ref asynchrone}), only one component can change its expression level from a state to its successor (\ie only one action is played at each step). In the following, if $\PHst \in \PHl$ is a state,
we call \emph{scenario from~$\PHst$}
any sequence of successive states from $\PHst$
obtained by playing some actions successively,
and we denote $\Sce(\PHst)$ the set of all scenarios from $\PHst$.

\begin{example}
In the PH model example of \pref{fig:ph}, if we consider an initial state $\PHst_0 = \PHstate{a_0, b_0, c_1, d_1, e_0}$ ---~or, knowing the sorts order, $\PHst_0 = (0, 0, 1, 1, 0)$~--- the only possible sequence of successive states from $\PHst_0$ in $\Sce(\PHst_0)$ is the following:
  \[(0, 0, 1, 1,0) \rightarrowtail (0, 0, 1, \underline{0}, 0) \rightarrowtail (\underline{1}, 0, 1, 0, 0) \rightarrowtail (1, \underline{1}, 1, 0, 0) \rightarrowtail (1, 1, 1, 0, \underline{1}).\]
\end{example}

A \emph{fixed point}, also called \emph{steady state},
is a state which has no successor,
as given in \pref{def:fixpoint}.
Such states have a particular interest as they denote states in which the model
stays indefinitely,
and the existence of several of these states denotes a switch in the dynamics~\cite{wuensche1998genomic}.

\begin{definition}[Fixed point]
\label{def:fixpoint}
  A state $\PHst \in \PHl$ is called a \emph{fixed point}
  (or equivalently \emph{steady state})
  if and only if it has no successors.
  In other words, $\PHst$ is a fixed point if and only if no action is playable in this state:
$$ \forall \PHfrappe{A}{b_j}{b_k} \in \PHa, A \not\subset \PHst \vee b_j \notin s \enspace. $$
\end{definition}

A finer and more general dynamical property consists in
the notion of \emph{attractors}.
Such a property, formally expressed in \pref{def:attractor},
states that a set of states that are connected and in which the system loops indefinitely (See \ref{fig:transition-graph}).

%states that starting from a given initial state, it is possible
%to reach a given goal, that is, a state that contains a process
%or a set of processes.
%Checking such a dynamical property is considered difficult
%as, in usual model-checking techniques,
%it is required to build (a part of) the state graph,
%which has an exponential complexity.

\begin{definition}[Cycle or Loop]
\label{def:cycle}
A set of states $\mathbf{C} = \bigcup\limits_{i=1}^{N} \PHst_{i}$ is called a \emph{cycle} of size $N$ if and only if $\forall \PHst \in \mathbf{C}, \PHst \in \Sce(\PHst) $.
\end{definition}

\Emna{du contenue à propos les cycles}
\Maxime{Trouver un autre terme ou une autre formulation à la place de long enough” pour dire : un scénario infini ou qui termine dans un stable state.}
Every long enough scenario of a network eventually converges to either a single state, or a set of states that cannot be escaped, called \emph{attractor}.

\begin{definition}[Attractor]
\label{def:attractor}
A set of $N$ states $\Delta = \bigcup\limits_{i=1}^{N} \PHst_{i}$ is called an \emph{attractor} of size $N$ if and only if $\Delta$ is a cycle and $\forall h \in \PHa $ playable in $\PHst \in \Delta$, $(\PHst \play h) \in \Delta$.
\end{definition}

By definition, if the system evolves and reaches any state of an attractor, it will infinitely cycle into the states of this same attractor.
We can also confuse an attractor of 1-size with a fixed point (\ie steady state).

\medskip
The aim of this paper is to focus on the resolution of issues related to the previous definitions:
we give algorithms enumerating all fixed points (\pref{sec:fixpoint})
and all attractors (\pref{sec:attractors}). We propose later to tackle the simulation of a PH model in ordor to verify which ones of these attractors are reachable.

\begin{figure}[h]
   \caption{\label{fig:transition-graph} A state transition graph of a PH model with 3 sorts (one sort with two levels $0$ or $1$ and two other sorts have three levels $0$, $1$ and $2$). This graph contains two attractors (size 2 and size 4) and one fixed point}
   \includegraphics{figures/transition-graph.pdf}
\end{figure}


\section{Answer Set Programming}
\Emna{A revoir et parler d'avantage des contraintes}
In this section, we briefly recapitulate the basic elements of Answer Set Programming (ASP) \cite{baral2003knowledge}, a  declarative language that proved efficient to address highly computational problems. An answer set program is a finite set of rules of the form:
\begin{equation}
\label{asprule}
  a_{0}\ \leftarrow \ a_{1},\ \ldots,\ a_{m},\ not\ a_{m+1},\ \ldots,\ not\ a_{n}
\end{equation}
where $n \ge m \ge 0$, $a_{0}$ is a propositional atom or $\bot$, all
$a_{1}, \ldots ,a_{n}$ are propositional atoms, and the symbol ``$not$'' denotes  negation as failure.
The intuitive reading of such a rule is that whenever $a_{1}, \ldots, a_{m}$
are known to be true and there is no evidence for any of the negated atoms $a_{m+1}, \ldots, a_{n}$ to be true, then $a_{0}$ has to be true as well.
If $a_{0} = \bot$, then the rule becomes a constraint (in which case the symbol $\bot$ is usually omitted).
As $\bot$ can never become true, if the right-hand side of a constraint is validated, it invalidates the whole answer set.

In the ASP paradigm, the search of solutions consists in computing the \emph{answer sets} of a given program.
An answer set for a program is defined by Gelfond and Lifschitz \cite{DBLP:conf/iclp/GelfondL88} as follows.
An \emph{interpretation} $I$ is a finite set of propositional atoms.
%An atom $a_i$ is \emph{true under $I$} if $a_i \in I$, and false otherwise.
A rule $r$ as given in \eqref{asprule} is \emph{true under~$I$} if and only if:
  \[\{a_1,\ \dots,\ a_{m}\} \subseteq I \wedge \{a_{m+1},\ \ldots,\ a_{n}\} \cap I = \emptyset \Rightarrow a_{0} \in I \enspace.\]
An interpretation $I$ is a \emph{model} of a program $P$ if each rule $r \in P$ is true under $I$.
Finally, $I$ is an answer set of $P$ if $I$ is a minimal (in terms of inclusion) model of $P^{I}$,
where $P^{I}$ is defined as the program that results from $P$ by deleting all rules that contain a negated atom that appears in $I$,
and deleting all negated atoms from the remaining rules.
Programs can yield no answer set, one answer set, or several answer sets.
To compute the answer sets of a given program, one needs a grounder (to remove variables from the rules) and a solver.
For the present work, we used \textsc{Clingo}\footnote{We used \textsc{Clingo} version 4.5.0: \url{http://potassco.sourceforge.net/}}~\cite{gebser2010incremental} which is a combination of both.

In section \pref{sec:defs} we detailed the dynamic properties (fixed points, cycles and attractors). How to verify them with the help of ASP is the subject of the rest of this paper.


\section{Fixed point enumeration}
\label{sec:fixpoint}
The study of basins of attraction provides an important understanding of the different behaviors of a Biological Regulatory Network (BRN)~\cite{wuensche1998genomic}.
Indeed, a system will always eventually end in a basin of attraction,
and this may depend on biological switches or other complex phenomena.
Here we focus on fixed points (also called stable states or steady states),
which are basins of attraction composed of only one state. \\

\subsection{Process Hitting translation in ASP}
Before any analysis of the network,
we first need to translate it into ASP\footnote{All programs, including this translation and the methods described in the following, are available online at: \url{https://github.com/EmnaBenAbdallah/verification-of-dynamical-properties_PH}}.
To do this we use the self-describing predicates
\texttt{sort}, \texttt{process} and \texttt{action} to define the sorts, processes and actions of the network, respectively.
\pref{ex:asp-ph} shows how a PH network is defined with these predicates.

\begin{example}[Representation of a PH model in ASP]
\label{ex:asp-ph}
The representation of the PH model of \pref{fig:ph} in ASP as the following:
\begin{lstlisting}
process("a", 0..1). process("b", 0..2). process("c", 0..1). %\label{ASPprocess1}
process("d", 0..1).  process("e", 0..2). %\label{ASPprocess2}
sort(X) :- process(X,_). %\label{ASPsort}
action("d",1,"d",1,0). action("d",0,"a",0,1). %\label{actions1}
action("a",1,"c",1,"b",0,1). action("a",1,"b",1,"e",0,1). %\label{actions2}
\end{lstlisting}
In \refll{ASPprocess1}{ASPprocess2} we create the list of processes corresponding to each sort.
For example the sort \texttt{"a"} has 2 processes numbered \texttt{0} and \texttt{1};
predicate ``\texttt{process("a", 0..1).}'' will in fact expand into the two following predicates:
\begin{lstlisting}[numbers=none]
process("a", 0). process("a", 1).
\end{lstlisting}
\Refl{ASPsort} enumerates every sort of the network from the previous information.
In ASP, a word starting with a capital letter is a variable,
and the underscore (``\texttt{\_}'') is a placeholder for any value.
Finally, all the actions of the network are defined straightforwardly in \refll{actions1}{actions2};
for instance, the predicate ``\texttt{action("d",0,"a",0,1).}'' represents the action
$\PHfrappe{d_0}{a_0}{a_1}$ and we use the same predicate to define a plural action: ``\texttt{action("a",1,"c",1,"b",0,1)}" to represent the action $\PHfrappe{a_1 \wedge c_1}{b_0}{b_1}$ means that when $a_1$ and $c_1$ are simultanory active, they can cooperate to change the level of $b$ from $0$ to $1$.
\end{example}

\subsection{Search of fixed points}
% du blabla sur les points fxes

\begin{algorithm}[h]
	\caption{Enumarate fixed points from an Automata Network}
	\label{alg:PH-fixpont}
	\begin{itemize}
		\item[] \textbf{INPUT}: Automata Network $PH=(\PHs,\PHl,\PHa)$
		\item[] \textbf{OUTPUT}: All the fixed points /* steady states or 1-size attractors
		\item[] \textbf{Initialize}: $\Delta \longleftarrow \varnothing$ /*  set of fixed points
		\item[] \textbf{Begin} \\
		
			\hspace{0.2cm}	\textbf{\textit{forall}} $\PHst_i \in \PHl$ \textbf{\textit{do}} 
				
				\begin{itemize}
					\item[] compute all playable actions $A_{\PHst_i}$ in $\PHst_i$
					
					\item[] \textbf{\textit{if}} $A_{\PHst_i} = \varnothing$ \textbf{\textit{then}} \\
						\hspace{0.5cm}  $\Delta \longleftarrow \Delta \cup \{\PHst_i\}$ \\
					 \textbf{\textit{end if}} 
				\end{itemize}		
			\hspace{0.2cm} \textbf{\textit{end forall}} \\		
			\hspace{0.2cm} \textbf{\textit{return}} $\Delta$		
	\end{itemize}
\end{algorithm}

The enumeration of fixed points requires to translate the definition of a steady state (given in \pref{def:fixpoint})
into logic rules for ASP program. Thus, first we browse all processes of this graph in order to generate all possible states,
that is, all possible combinations of processes by choosing only one process from each sort (\refl{selectprocesses}).
\begin{lstlisting}
1 {selectProc(A,I) : process(A,I)} 1 :- sort(A). %\label{selectprocesses}
\end{lstlisting}
The previous lines form a \emph{cardinality rule} that creates as many potential answer sets as the number of possible states
to take into account.
The next step consists in filtering out any state that is not a \emph{fixed point},
or, in other words, eliminating all candidate answer sets in which an action could be played. For this, we use a constraint:
any solution that satisfies the body of this constraint will be removed from the answer set.
Regarding our problem, a state is eliminated if there exists an action between two selected processes (\refll{constraintFix}{constraintFix2}). The first constraint \refl{constraintFix} treats the action with only one hitter and the second \refl{constraintFix2} the actions with $2$ hitters. We developed so a program that takes into account the maximum indegree of the model and generate constraints as much as necessary.
\begin{lstlisting}
:- action(A,I,B,J,_), selectProc(A,I), selectProc(B,J), A!=B. %\label{constraintFix}
:- action(A1,I1,A2,I2,B,J,_), selectProc(A1,I1),selectProc(A2,I2), 
      selectProc(B,J), A!=B. %\label{constraintFix2}
\end{lstlisting}
Finally, the fixed points of the considered model are exactly the states represented in each remaining answer,
described by the atoms \texttt{selectProc(A,I)}.

\begin{example}[Fixed points enumeration]
The PH model of \pref{fig:ph} contains $5$ sorts:
$a$, $b$, $c$ and $d$ have $2$ processes and $b$ has 3; therefore, the whole model has $2*2*2*3*3 = 72$ states (whether they can be reached or not from a given initial state).
We can check that this model contains exactly $5$ fixed points: $\PHstate{a_1, b_0, c_0, d_0, e_1}$, $\PHstate{a_1, b_2, c_0, d_0, e_1}$, $\PHstate{a_1, b_1, c_0, d_0, e_1}$, $\PHstate{a_1, b_0, c_0, d_0, e_0}$, $\PHstate{a_1, b_2, c_0, d_0, e_0}$.
Indeed, there is no action between each two processes contained in this state so no execution is possible from these. In this example, no other state verifies this property.

If we execute the ASP program detailed above (\refll{selectprocesses}{constraintFix2}),
alongside with the description of the PH model given in \pref{ex:asp-ph} (\refll{ASPprocess1}{actions2}),
we obtain $8$ answer sets that match the expected result:
\begin{changemargin}{-2cm}{-1cm}
%\scriptsize{ 
\begin{lstlisting}[numbers=none]
Answer: 1
selectProc("a",1) selectProc("b",0) selectProc("c",0) selectProc("d",0) selectProc("e",1)
Answer: 2
selectProc("a",1) selectProc("b",2) selectProc("c",0) selectProc("d",0) selectProc("e",1)
Answer: 3
selectProc("a",1) selectProc("b",1) selectProc("c",0) selectProc("d",0) selectProc("e",1)
Answer: 4
selectProc("a",1) selectProc("b",0) selectProc("c",0) selectProc("d",0) selectProc("e",0)
Answer: 5
selectProc("a",1) selectProc("b",2) selectProc("c",0) selectProc("d",0) selectProc("e",0)
\end{lstlisting}
%}
\end{changemargin}
\end{example}

\section{Attractors }
\label{sec:attractors}
%\subsection{Intuitive idea}

In the previous section we showed a method to enumerate all fixed points.
In the following, we focus on the more complex kind of minimal trap domains: attractors.
We describe here how to obtain some or all attractors of a fixed size;
obtaining all attractors of any size
can be tackled by gradually increasing the considered size.

The computational method to enumerate all attractors of size $N$ in a PH model falls into three stages:
\begin{enumerate}
  \item Enumerating all sub-State Transition Graphs (sub-STGs)
    with $N$ transitions,
  \item Filtering out all of these sub-STG that are not cyclic,
  \item Checking, for each state of the remaining cycles,
    if all the successor states are also part of this cycle.
\end{enumerate}
Once all steps are passed, the only remaining sub-STGs are attractors of size $N$,
because they consist of cycles that cannot be escaped.

\Maxime{Décrire brièvement l'algo ?}
Once again, for the sake of clarity,
the pseudocode for the imperative equivalent of our method is given as Algorithm \ref{alg:PH-attractor}.
However, our method (including all three steps mentioned above) has been implemented in a purely ASP fashion.
Finally, we note that this approach allows 

%Faire un algo
\begin{algorithm}[p]
	\caption{Enumeration of the attractors of size $N$ of a PH}
	\label{alg:PH-attractor}
	\begin{itemize}
		\item[] \textbf{INPUT:} A network $PH=(\PHs,\PHl,\PHa)$, an integer $N$ \quad // $N \geq 2$
		\item[] \textbf{OUTPUT:} All the attractors of size $N$
		\item[] \textbf{Initialize:} $\Delta_N \longleftarrow \varnothing$ \quad // Will be the set of the attractors of size $N$
		\item[] \textbf{Begin:} \\
		
			\hspace{0.2cm}	\textbf{\textit{forall}} $\PHst_i \in \PHl$ \textbf{\textit{do}} 
				
				\begin{itemize}
					\item[] Compute the $N$ successors: $\PHst_i \rightarrowtail ... \rightarrowtail \PHst_N$ \\ Let $\Delta_N^i=$\{$\PHst_j \in \PHl, i \leq j \leq N$\} %be the set of these states
					
					\item[] \textbf{\textit{if}} $\PHst_i = \PHst_N$ \textbf{\textit{then}} \\
						\hspace{0.7cm}  $loop(\Delta_N^i) \longleftarrow True$ \\
					 \textbf{\textit{else}} \\
					 	\hspace{0.7cm}  $loop(\Delta_N^i) \longleftarrow False$ \\
					 \textbf{\textit{end if}} 
					 
					\item[] \textbf{\textit{if}} $loop(\Delta_N^i) = True$ \textbf{\textit{then}} \\
					
%						\begin{itemize}
							\hspace{0.7cm} $attractor(\Delta_N^i) \longleftarrow True$	\\
							
							\hspace{0.7cm} \textbf{\textit{forall}} $\PHst_j \in \Delta_N^i$ \textbf{\textit{do}} \\
							
							\hspace{1.2cm} \textbf{\textit{if}} $\exists$ $\Sce(\PHst_j) \notin  \Delta_N^i$ \textbf{\textit{then}} \\ \hspace{1.7cm} $attractor(\Delta_N^i) \longleftarrow False$ \\
							\hspace{1.2cm} \textbf{\textit{end if}} 
							
							\textbf{\textit{else}} \\
							\hspace{0.7cm} $attractor(\Delta_N^i) \longleftarrow False$ \\
%						\end{itemize}		
							\textbf{\textit{end if}} 
														
					\item[] \textbf{\textit{if}} $attractor(\Delta_N^i) = True$ \textbf{\textit{then}}\\
						\hspace{0.7cm} \textbf{\textit{forall}} $\PHst_j, \PHst_k \in \Delta_N^i, j \neq k$ \textbf{\textit{do}} \\								
								\hspace{1.2cm} \textbf{\textit{if}} $loop(\Delta_k^j) = True$ \textbf{\textit{then}}\\
									\hspace{1.7cm} $complexLoop(\Delta_N^i) \longleftarrow False$ \\
									\hspace{1.7cm} $x \longleftarrow j+1$ \\									
									\hspace{1.7cm} \textbf{\textit{while}} $x \leq k-1 $ \textbf{\textit{and}} $complexLoop(\Delta_N^i) = False$ \textbf{\textit{do}} \\	
									
										\hspace{2.1cm} \textbf{\textit{if}} $loop(\Delta_{k+x}^{j+x}) = False$ \textbf{\textit{then}}\\
											\hspace{2.5cm} $complexLoop(\Delta_N^i) \longleftarrow True$ \\
										\hspace{2.1cm} \textbf{\textit{end if}} \\
									\hspace{1.7cm} \textbf{\textit{end while}} \\
								\hspace{1.2cm} \textbf{\textit{end if}} \\
						\hspace{0.7cm} \textbf{\textit{end forall}} 	\\

					\hspace{0.7cm} \textbf{\textit{if}} $complexLoop(\Delta_N^i) = False$ \textbf{\textit{then}}\\
						\hspace{1.2cm}  $attractor(\Delta_N^i) \longleftarrow False$\\
					\hspace{0.7cm} \textbf{\textit{end if}} 						
						
					\textbf{\textit{end if}} \\

					\item[] \textbf{\textit{if}} $attractor(\Delta_N^i) = True$ \textbf{\textit{then}}\\
					\hspace{0.5cm}  $\Delta_N \longleftarrow \Delta_N \cup \{ \Delta_N^i \}$ \\
					\textbf{\textit{end if}} 

				\end{itemize}		
			\hspace{0.2cm} \textbf{\textit{end forall}} \\		
			\hspace{0.2cm} \textbf{\textit{return}} $\Delta_N$		
	\end{itemize}
\end{algorithm}

% Mettre du code ASP + explication
\subsection{Cycles enumeration}
The approach that we present here first searches for all sub-graphs with a given number $N$ of transitions %(i.e path of length $N$)
in the network.
For this, we once again start from every possible initial state in the model (\refl{selectprocesses}). 

The evolution of a model is then computed within a limited number $N$ of steps. In an ASP program, it is possible to instantiate constants whose values will be defined by the user at each execution. Here, we use \texttt{n} as a constant that represents the number of states we want to consider
inside the predicate \texttt{state(0..n)}.
For example, \texttt{state(0..5)} will compute every 6 first successive states, including the initial state.

Because the dynamics of a PH is solely described by its actions, identifying the future states requires to first identify the playable actions for each state.
We recall that an action is playable in a state when all its hitter processes and target process are active in this state (see \pref{def:playableAction}).
% Therefore, we define an ASP predicate $\texttt{playable(action(A}_\texttt{1}\texttt{,I}_\texttt{1}\texttt{,...,A}_\texttt{k}\texttt{,I}_\texttt{k}\texttt{,B,J,K),S)}$ which is true when the processes  ${\texttt{A}_\texttt{1}}_{\texttt{I}_\texttt{1}}\texttt{,...,}{\texttt{A}_\texttt{k}}_{\texttt{I}_\texttt{k}}$ and $\texttt{B}_\texttt{J}$ are active at state \texttt{S}.
Therefore, we define an ASP predicate $\texttt{playable(action(A1,I1,...,Ak,Ik,B,J,K),S)}$,
whose construction is not detailed here,
which is true when the processes $\texttt{A1}_{\texttt{I1}}$, ..., $\texttt{Ak}_{\texttt{Ik}}$ and $\texttt{B}_\texttt{J}$ are active in state \texttt{S}.

Then, all the possible evolutions of the model can be enumerated by a cardinality rule such as the one of \refl{e2a}:
this line creates a set of as many predicates as there are possible simulations from the considered initial state,
thus reproducing all possible branchings in the dynamics of the model, in the form of as many new potential answer sets.
It is also needed to enforce the strictly asynchronous dynamic
which forces that exactly one process can change between two states (thus maintaining the non-determinism).
We use the constraint of \refl{e2b} to remove all scenarios where two or more actions have been played simultaneously between two states.
Thus, the remaining scenarios contained in the answer sets all strictly follow
the asynchronous dynamics defined in \pref{sec:defs}.
We witness that the active process of a given sort \texttt{B} has changed between two consecutive states \texttt{S} and \texttt{S+1} with the predicate \texttt{change(B,S+1)} of \refll{e3a}{e3b}
depending on the chosen action to be played.
For this, it is once again necessary to duplicate this rule depending on the maximal number of hitters in all actions in the model,
as we did for the rules of \refll{constraintFix}{constraintFix2}.
Finally, all active processes of the new state of step \texttt{S+1}
are gathered into a predicate \texttt{instate}.
This predicate endorses the fact that one process has evolved (\refl{e4})
while all the processes from all the other sorts
remain identical in their condition of the previous state (\refll{e5}{e5a}).

\begin{lstlisting}
1 {play(Action,S)} 1 :- playable(Action,S), state(S). %\label{e2a} 
:- 2 {play(Action,S)}, state(S). %\label{e2b}
change(B,S+1) :- play(action(_,_,B,_,_),T), state(S). %\label{e3a}
change(B,S+1) :- play(action(_,_,_,_,B,_,_),T), state(S). %\label{e3b}
instate(activeProc(B,K),S+1) :- play(action(_,_,B,_,K),S), state(S).%\label{e4}
instate(activeProc(B,K),S+1) :- instate(activeProc(B,K),S),%\label{e5}
        not change(B,S+1), state(S).%\label{e5a}
\end{lstlisting}

The overall result of the pieces of program presented in this subsection
(\refl{selectprocesses} and \refll{e2a}{e5a})
is a collection of answer sets,
with one answer for each possible simulation in $N$ time steps,
starting from any initial state.
As a result, it reproduces all sub-STG of size $N$.
%%%%%%%%%%%

Once a sub-STG is found, it remains to check whether it is a cycle (\ie a loop) or not.
To do so, we need a predicate \texttt{repeter(P,S1,S2)} (\refll{repeter1}{repeter2}) which is true for each process active in both \texttt{S1} and \texttt{S2}.
Predicate \texttt{getNbrRepetition(X,S1,S2)} (\refll{getRep1}{getRep2}) then computes the number \texttt{X} of identical active processes between two states \texttt{S1} and \texttt{S2}.
\begin{lstlisting}
repeter(P,S1,S2) :- selectProcSt(P,S1), selectProcSt(P,S2), %\label{repeter1}
        state(S1), state(S2). %\label{repeter2}
getNbrRepetition(X,S1,S2) :- X={repeter(F,S1,S2)}, S1!=S2, %\label{getRep1}
        state(S1), state(S2). %\label{getRep2}
\end{lstlisting}

At this point, determining if there is a loop between \texttt{S1} and \texttt{S2}
only requires to compare the value \texttt{X} to the total number of sorts in the model:
if both are equal, it means that \texttt{S1} and \texttt{S2} are identical states.
In our case, we only need to compare the first (\texttt{S1=0}) and last (\texttt{S2=n}) states of each sub-STG.
Thus, we finally eliminate all sub-STG that are not cycles with the constraint of \refl{loop},
where the predicate \texttt{loop} is true when both states are identical.
\begin{lstlisting}
:- not loop(0, n).  %\label{loop}
\end{lstlisting}

\subsection{Attractors enumeration}
Since each state of a STG of a PH has various next states (due to the non-deterministic dynamics), a cyclic sub-STG is not necessarily an attractor.
Indeed, some actions may allow the dynamics to escape the cycle;
in this case, the cycle would not even be a trap domain.
Thus, another check is required to filter out all remaining cycles
that can be escaped (and are therefore not attractors).
Once again, this filtering is performed with a constraint,
which is the most suitable solution.
We thus need to describe the behavior that we do not wish to observe
---~that is, the escaping of a cycle~---
in order to use in in the constraint in the end.

In order to create the desired constraint,
it is necessary to compute all playable actions in each state $\texttt{S}$ comprised between the states \texttt{0} and \texttt{n}.
For that we propose a predicate \texttt{candidatePlayable}
that is similar to the predicate \texttt{playable} used in \refl{e2a}.
This is performed by selecting a playable action in any state \texttt{S}
(\refll{comptPlayAct}{comptPlayActBis})
which must be different from the action played to create the cycle
by creating a transition from state \texttt{S} to state \texttt{S+1}
(\refll{candidatPlayActBis}{candidatPlayAct}).
\begin{lstlisting}
firstCandidatePlayable(action(A,I,B,J,K),S) :- action(A,I,B,J,K), %\label{comptPlayAct}
        selectProcSt(activeProc(B,J),S),
        selectProcSt(activeProc(A,I),S),
        loop(0, n), state(S), S > 0, S < n. %\label{comptPlayActBis}
candidatePlayable(Action1,S) :- firstCandidatePlayable(Action1,S), %\label{candidatPlayActBis}
        playable(Action2,S), Action1 != Action2. %\label{candidatPlayAct}
\end{lstlisting}

After finding the candidate playable actions in each state of the cycle, it remains to verify if the resulting states are already part of the cycle or if they consist in new states.
Indeed it is possible to have other playable actions in the cycle,
which is described by the predicate \texttt{playableInCycle}
(\refll{playIncyc1}{playIncyc3}).
Such actions are simply “shortcuts” to other states in the same cycle
and do not contradict the definition of an attractor.
Here, we are rather interested in the existence of a playable action
that would allow to escape the cycle
(\refll{existActOut0}{existActOut1}).
Provided that we don't want this case to happen,
the constraint is simply expressed by line \refl{existOut}.
\begin{lstlisting}
playableInCycle(action(A,I,B,J,K),S) :- %\label{playIncyc1}
        candidatePlayable(action(A,I,B,J,K),S),
        selectProcSt(activeProc(B,K),T), T >= 0, T <= n, T != S+1, %\label{playIncyc2}
        getNbrRepetition(X,S,T), getNbreSorts(X+1), state(T). %\label{playIncyc3}
existActionOut :- candidatePlayable(Action,S), %\label{existActOut0}
        not playableInCycle(Action,S). %\label{existActOut1}
:- existActionOut.  %\label{existOut}
\end{lstlisting}

At this point, we managed to enumerate all attractors in a PH
which size is $N$ or a divisor of $N$.
Indeed, as is, our method may include repeated cycles in the results;
for instance, an attractor
$\PHst_1 \rightarrowtail \PHst_2 \rightarrowtail \PHst_1$ of size $N=2$
will be listed among the
attractors of size $N=4$ if repeated twice:
$\PHst_1 \rightarrowtail \PHst_2 \rightarrowtail \PHst_1 \rightarrowtail \PHst_2 \rightarrowtail \PHst_1$.
We thus wish to filter out all repeated cycles from the results.
For this, the predicate \texttt{loopX(S1,S2,X)} of \refl{loopX} becomes true if there is a loop between
the \texttt{X}\textsuperscript{th} successor of \texttt{S1}
and the \texttt{X}\textsuperscript{th} successor of \texttt{S2}
in the considered succession of states.
This predicate is a clue that the same loop may have been traversed twice.
Moreover, in \refl{notloopX}, \texttt{notloopX(S1,S2)} states that there exists at least one \texttt{X}\textsuperscript{th} successor of \texttt{S1} that does not loop back to the \texttt{X}\textsuperscript{th} successor of \texttt{S2}.
Finally, the constraint in \refl{constraintloopX} eliminates all cases where a loop is repeated by using the previously presented predicates.
\begin{lstlisting}
loopX(S1,S2,X) :- loop(S1+X, S2+X), 0<=X, X<=S2-S1, state(S1;S2;X). %\label{loopX}
notloopX(S1,S2) :- not loopX(S1,S2,X), 0<=X, X<=S2-S1, state(S1;S2;X). %\label{notloopX}
:- loop(S1,S2), loop(S1, S2+(S2-S1)), not notloopX(S1,S2). %\label{constraintloopX}
\end{lstlisting}

The problem of finding attractors in a BRN is NP-hard.
However, the solvers of ASP (namely, \textsc{Clingo} in our case)
are specialized in tackling such complex problems.
Next section will be dedicated to the results
of several computational experiments that we performed on biological networks,
and we show that our implementation in ASP can handle
even large systems and return its results in only a few seconds in some cases.



%
%
%Clearly, all states between any two occurrences of the last state belong to a loop. A loop corresponds to an attractor. We mark all attractor’s states. In the following iterations, we will only search for paths in which the last state is not marked.
%Until at least one attractor remains unmarked, we can find a path of any length since we can cycle in an attractor forever. However, once all attractors are identified and marked, we will only be able find paths which are shorter than a given length (at most the diameter of the STG). So, when we search for a path of some length k and it does not exist, this means that all attractors are already identified and the algorithm can terminate.
%If a path of length k does exist and it is loop-free, we double k and continue the search for a path of the new length.
%We illustrate the algorithm on the example of the Boolean network in Figure 1. The algorithm starts from searching for a path of length $N = 3$


% Pas sure de laisser cette partie on peut la mettre dans un autre article + une étude approffondie (bifurcation ou autre...)
%\section{Reachability of attractors }
%\label{sec:reach-attractor}
%\input{parts/reach-attractors}

\section{Results }
\label{sec:results}
% Section des résultat: Application sur des réseaux biologiques

In this section, we exhibit several experiments conducted on biological networks.
We detail the results or our implementation
on a 4-components model of the lambda phage,
and recapitulate the results and performance of some benchmarks
on other models up to 28 components.
In general, the time performances are good and
the overall results seem to support the use of ASP for the verification
of formal properties or the enumeration of special constructs
on biological systems.

All experiments were run on a desktop PC
with a Pentium VII 3 GHz processor and 16 GB memory.

\subsection{Lambda phage model}
\begin{figure}[h]
  \includegraphics{figures/lampdaphage-thomas.pdf}
  \includegraphics{figures/lampdaphage-STG.pdf}
  \caption{\label{fig:lambda-graph} 
    Descriptions of the 4-components lambda phage model used for our experiments.
    The left figure presents the regulatory graph used with positive and negative regulations represented by ’$+$’ and ’$-$’ symbols.
    The right figure is the corresponding state transition graph (STG).% colored according to strongly connected components. 
    We can observe two attractors: $(2000)$ is an attractor of size $1$, \ie a fixed point and $\{(0300), (0200)\}$ is an attractor of size $2$.
    Figures were taken from \cite{thieffry1995dynamical,chaouiya2008petri}.
  }
\end{figure}

We first conducted detailed experiments on a 4-components model of the bacteriophage lambda system described in \pref{fig:lambda-graph} \cite{thieffry1995dynamical}.
This network was modeled as a PH network containing 4 sorts, 48 processes and 49 actions.
Its State-Transition Graph (STG) comprises \Emna{???} different states.

An analytic study of the minimal trap domains on this small network
allows to find the following structures:
\begin{itemize}
  \item exactly one fixed point: $\PHstate{CI_2, CII_0, Cro_0, N_0}$
  \item exactly one cyclic attractor: $\{ \PHstate{CI_0, CII_0, Cro_2, N_0}$, $\PHstate{CI_0, CII_0, Cro_3, N_0} \}$
\end{itemize}
We note that the attractor happens to be of size 2.

The implementation of the method described in \pref{sec:fixpoint} and \pref{sec:attractors}
of course returns the same results.
The output of the fixed points enumeration program is straightforward,
as it contains exactly one answer set (corresponding to the only existing fixed point) which describes the only stable state of the model:
\begin{lstlisting}[numbers=none]
Answer: 1
selectProc("CI",2) selectProc("CII",0)
selectProc("Cro",0) selectProc("N",0)
\end{lstlisting}
Performing the search for size-2 attractors also returns exactly one answer set corresponding to the only attractor of that size.
In the output, each state belonging to this attractor is labeled by a number (\texttt{0} and \texttt{1}) allowing to enumerate all active processes in both states:
\begin{lstlisting}[numbers=none]
Answer: 1
selectProcSt(activeProc("CI",0),0) selectProcSt(activeProc("CII",0),0)
selectProcSt(activeProc("Cro",2),0) selectProcSt(activeProc("N",0),0) 
selectProcSt(activeProc("CI",0),1) selectProcSt(activeProc("CII",0),1)
selectProcSt(activeProc("Cro",3),1) selectProcSt(activeProc("N",0),1)
\end{lstlisting}
Moreover, executing out implementation for $N>2$ returns no results,
which is expected because there exists no attractor of size strictly bigger than two, and we excluded repeated loops from the results (therefore the already found size-2 attractor is not found for $N=4$).
On this small network, all results are computed in less than a tenth of a second.

\subsection{Benchmarks}
In the following, we propose some additional benchmarks to demonstrate
the capabilities of our implementation.
We do not give the detail of the results of these experiments
but rather focus on the size of the outputs and the computation times.
We used several preexisting multi-valued networks inspired from real organisms
and found in the literature:
a bacteriophage lambda \cite{thieffry1995dynamical},
a hedgehog signaling pathway (\Emna{à revoir la réf} \cite{stecca2010context}),
an mTOR pathway in survival mechanisms \cite{javle2010inhibition} (\Emna{en cours de traduction})
and a mammalian signaling pathway induced by the Epidermal Growth Factor (EGF) and Tumour Necrosis Factor alpha (TNF$\alpha$) \cite{chaouiya2013sbml}.
The ASP description of these networks in PH form has been realized manually
from the data given in the corresponding papers.
Results of these benchmarks\benchmarksfootnote{} are given in \pref{tab:models}.

\begin{table*}[ht]
\begin{center}
\noindent%
%\begin{tabular}{| l| c | c | c ||>{\columncolor{verylightgray}} c | c ||>{\columncolor{verylightgray}}c | c | c ||}
\begin{tabular}{| l || c | c | c || c | c || c | c | c |}
\hline
  \multicolumn{1}{| l ||}{\textbf{Models}} &
    \multicolumn{3}{c||}{\textbf{PH model}} &
    \multicolumn{2}{c||}{\textbf{Fixed points}} &
    \multicolumn{3}{c|}{\textbf{Attractors}}\\
\hline
  &
%    sorts & processes & states &
    $\card{\PHs}$ & $\card{\bigcup_a \PHl_a}$ & $\card{\PHl}$ &
    $\Delta t$ (s) & \#results &
    $N$ & $\Delta t$ (s) & \#results \\
\hline
\hline
%  TTR  & 12 & 42  & $2^{19}$ & 0.004 & 0 & 0.004 & k & 0\\
%\hline
%  ERBB & 42 & 152 & $2^{70}$ & 0.017 & 3 & 0.004 & k & 0\\
%\hline
%  TCR & 54 & 156 & $2^{73}$ & 0.021 & 1 & 0.004 & k & 0\\
%\hline  
  \multirow{3}{*}{\begin{minipage}{4em}\textbf{lambda phage}\end{minipage}} &
    \multirow{3}{*}{4} & \multirow{3}{*}{11} & \multirow{3}{*}{48} &
    \multirow{3}{*}{0.003} & \multirow{3}{*}{1} &
    2 & 0.019 & 1\\
    & & & & & & 4 & 0.033 & 0\\
    & & & & & & 20 & 0.470 & 0\\

\hline
  \multirow{3}{*}{\textbf{hedgehog}} &
    \multirow{3}{*}{5} & \multirow{3}{*}{11} & \multirow{3}{*}{48} &
    \multirow{3}{*}{0.003} & \multirow{3}{*}{2} &
    2 & 0.017 & 0\\
    & & & & & & 6 & 0.050 & 0\\
    & & & & & & 20 & 0.319 & 0\\

\hline
  \multirow{3}{*}{\textbf{mTOR}} &
    \multirow{3}{*}{6} & \multirow{3}{*}{12} & \multirow{3}{*}{$2^6$} &
    \multirow{3}{*}{0.009} & \multirow{3}{*}{1} &
    2 & 0.019 & 0\\
    & & & & & & 8 & 0.101 & 0\\
    & & & & & & 16 & 0.397s & 0\\

\hline
  \multirow{2}{*}{\textbf{egf-tnf}} &
    \multirow{2}{*}{28} & \multirow{2}{*}{56} & \multirow{2}{*}{$2^{28}$} &
    \multirow{2}{*}{0.005} & \multirow{2}{*}{2} &
    2 & 0.089 & 0\\
    & & & & & & 8 & 9.192 & 1\\
% & & & & & & xxx & xx & x\\
\hline
\end{tabular}
\Maxime{--- Je pense qu'il vaudrait vraiment mieux donner le temps total pour trouver l'ensemble des attracteurs (on fait bien noter papier sur l'énumération). On peut donner les deux temps (énumération complète et premier attracteur trouvé).}

\Maxime{--- On peut encore gagner de la place horizontalement en insérant les résultats des points fixes en haut de ceux des attracteurs.}

\vspace*{4pt}
\caption{\label{tab:models}%
  Brief description of the models used in our benchmarks
  and the results of our fixed points and attractors enumerations.
  The successive lines sum up the information regarding, respectively,
  the bacteriophage lambda \cite{thieffry1995dynamical},
  the hedgehog signaling pathway \cite{stecca2010context},
  the survival mechanisms pathway (mTOR) \cite{javle2010inhibition}
  and the EGF- and TNF$\alpha$-induced mammalian signaling pathway (egf-tnf) \cite{chaouiya2013sbml}.
  For each of them, this table gives
  %the number of biological components in the original representation,
  the number of sorts and processes in the PH representation,
  and the number of states in the corresponding STG;
  the computation time and number of results of the fixed points enumeration;
  the computation time and number of results of the attractors enumeration for several sizes ($N$).
}
\end{center}
\end{table*}


\section{Conclusion and future directions}
\label{sec:ccl}
\Maxime{Début à revoir.}
This paper presents a new logical approach to compute attractors
in a particular class of automata networks.
This class encompasses qualitative asynchronous biological networks,
making all results especially applicable to the widespread Thomas modeling.

The point of the method developed in this article is the computation and
the enumeration of the attractors of such a network.
We present a particular method for attractors of size 1 (also called fixed points or steady states)
and a more general one for all attractors of given size $N \geq 2$.
The originality of our work consists in using ASP, a powerful declarative programming paradigm.
Thanks to the encoding we introduced, and the powerful heuristics developed in modern solvers, we are able to tackle the enumeration of fixed points, cycles and attractors of medium-sized models.
The major benefit of such a method is to get an exhaustive enumeration of all potential states while still being tractable for models with dozens of interacting components.
Identifying attractors in large biological networks is a big challenge because it gives an insight to the long-term behavior of biological systems,
and we hope our work helps open new ways and tools to explore this field.

\Maxime{Il faudrait peut-être étoffer un peu la conclusion.}

\Maxime{Et ajouter des perspectives.}

%Perspective: The attractor inverse problem, which involves construct- ing BNs possessing a given attractor structure, is important to network inference from steady-state data. Most microarray- based gene-expression studies do not involve controlled tem- poral experimental data; rather, it is assumed that the data result from sampling from the steady state. Under the BN model, this means that the data come from the attractors.

%In this paper, we proposed a new logical approach to address some dynamical properties of Process Hitting models.
%The originality of our work consists in using ASP, a powerful declarative programming paradigm.
%Thanks to the encoding we introduced, we are not only able to tackle the enumeration of fixed points but also to check reachability properties.
%The major benefit of such a method is to get an exhaustive enumeration of all corresponding paths while still being tractable for models with dozens of interacting components.
%
%One of the perspectives of our work is to extend the set of models on which our approach could be applied. We can consider the addition of priorities or neutralizing edges, or tackle other representations,
%such as Thomas modeling~\cite{Thomas73}.
%However, the range of the analysis can also be enriched,
%by searching the set of initial states
%allowing to reach a given goal instead of the other way around,
%or extending the method to universal properties (like the $\mathsf{AF}$ operator in CTL).



%%%%%%%%%%%%%%%%%%%%%%%%%%%%%%%%%%%%%%%%%%%%%%%%%%%%%%%%%%%%%
%%                  The Bibliography                       %%
%%                                                         %%
%%  Bmc_mathpys.bst  will be used to                       %%
%%  create a .BBL file for submission.                     %%
%%  After submission of the .TEX file,                     %%
%%  you will be prompted to submit your .BBL file.         %%
%%                                                         %%
%%                                                         %%
%%  Note that the displayed Bibliography will not          %%
%%  necessarily be rendered by Latex exactly as specified  %%
%%  in the online Instructions for Authors.                %%
%%                                                         %%
%%%%%%%%%%%%%%%%%%%%%%%%%%%%%%%%%%%%%%%%%%%%%%%%%%%%%%%%%%%%%

% if your bibliography is in bibtex format, use those commands:
\bibliographystyle{bmc-mathphys} % Style BST file (bmc-mathphys, vancouver, spbasic).
\bibliography{biblio}      % Bibliography file (usually '*.bib' )
% for author-year bibliography (bmc-mathphys or spbasic)
% a) write to bib file (bmc-mathphys only)
% @settings{label, options="nameyear"}
% b) uncomment next line
%\nocite{label}


\end{document}
