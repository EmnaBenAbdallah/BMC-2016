\Maxime{Changer peut-être un peu cette entrée en matière pour varier par rapport à la version précédente.}

The first aspect of our present work is the enumeration of a special kind of basins of attraction: fixed points (also called stable states or steady states) which are composed of only one state.
They can be studied separately from attractors because their enumeration follows a different pattern which is more specific to this problem.

\subsection{Process Hitting translation in ASP}
Before any analysis of the network,
we first need to translate it into ASP \benchmarksfootnote.
To do this we use the self-describing predicates
\texttt{sort}, \texttt{process} and \texttt{action} to define the sorts, processes and actions of the network, respectively.
\pref{ex:asp-ph} shows how a PH network is defined with these predicates.

\begin{example}[Representation of a PH model in ASP]
\label{ex:asp-ph}
The representation of the PH model of \pref{fig:ph} in ASP as the following:
\begin{lstlisting}
process("a", 0..1). process("b", 0..2). process("c", 0..1). %\label{ASPprocess1}
process("d", 0..1).  process("e", 0..2). %\label{ASPprocess2}
sort(X) :- process(X,_). %\label{ASPsort}
action("d",1,"d",1,0). action("d",0,"a",0,1). %\label{actions1}
action("a",1,"c",1,"b",0,1). action("a",1,"b",1,"e",0,1). %\label{actions2}
\end{lstlisting}
In \refll{ASPprocess1}{ASPprocess2} we create the list of processes corresponding to each sort;
for example, the sort \texttt{"a"} has 2 processes numbered \texttt{0} and \texttt{1} and
predicate ``\texttt{process("a", 0..1).}'' will in fact expand into the two following predicates:
\begin{lstlisting}[numbers=none]
process("a", 0). process("a", 1).
\end{lstlisting}
\Refl{ASPsort} enumerates every sort of the network from the previous information.
In ASP, a word starting with a capital letter is a variable,
and the underscore (``\texttt{\_}'') is a placeholder for any value.
Finally, all the actions of the network are defined straightforwardly in \refll{actions1}{actions2};
for instance, the predicate ``\texttt{action("d",0,"a",0,1).}'' represents the action
$\PHfrappe{d_0}{a_0}{a_1}$ and we use the same predicate to define an action with several hitters, for instance: ``\texttt{action("a",1,"c",1,"b",0,1)}" to represent the action $\PHfrappe{a_1 \wedge c_1}{b_0}{b_1}$ meaning that when $a_1$ and $c_1$ are simultaneously active, they can cooperate to change the level of $b$ from $0$ to $1$.
\end{example}

\subsection{Search of fixed points}
% du blabla sur les points fxes

\Maxime{Faire référence à l'algo 1 et expliquer :
\begin{itemize}
  \item Qu'on l'ajoute pour les gens plus à l'aise avec l'impératif,
  \item Qu'il permet d'avoir une idée de la complexité du problème étudié,
  \item Mais qu'on ne le donne pas à ASP.
\end{itemize}
}

\begin{algorithm}[h]
	\caption{Enumarate fixed points from an Automata Network}
	\label{alg:PH-fixpont}
	\begin{itemize}
		\item[] \textbf{INPUT}: Automata Network $PH=(\PHs,\PHl,\PHa)$
		\item[] \textbf{OUTPUT}: All the fixed points // steady states or 1-size attractors
		\item[] \textbf{Initialize}: $\Delta \longleftarrow \varnothing$ //  set of fixed points
		\item[] \textbf{Begin} \\
		
			\hspace{0.2cm}	\textbf{\textit{forall}} $\PHst_i \in \PHl$ \textbf{\textit{do}} 
				
				\begin{itemize}
					\item[] compute all playable actions $A_{\PHst_i}$ in $\PHst_i$
					
					\item[] \textbf{\textit{if}} $A_{\PHst_i} = \varnothing$ \textbf{\textit{then}} \\
						\hspace{0.5cm}  $\Delta \longleftarrow \Delta \cup \{\PHst_i\}$ \\
					 \textbf{\textit{end if}} 
				\end{itemize}		
			\hspace{0.2cm} \textbf{\textit{end forall}} \\		
			\hspace{0.2cm} \textbf{\textit{return}} $\Delta$		
	\end{itemize}
\end{algorithm}

The enumeration of fixed points requires to translate the definition of a steady state (given in \pref{def:fixpoint})
into logic rules for ASP program. Thus, we first browse all processes of this graph in order to generate all possible states,
that is, all possible combinations of processes by choosing only one process from each sort (\refl{selectprocesses}).
\begin{lstlisting}
1 {selectProc(A,I) : process(A,I)} 1 :- sort(A). %\label{selectprocesses}
\end{lstlisting}
This line forms a \emph{cardinality rule} that creates as many potential answer sets as the number of possible states
to take into account.
Cardinality rules are very useful to enumerate candidate answer sets,
and they allow to tune how many atom of each kind is included in every set.
In this particular case, we stated that a state (\texttt{selectProc})
is created by considering exactly one \texttt{process} for each \texttt{sort}.

The next step consists in filtering out any state that is not a fixed point,
that is, in this case, eliminating all candidate answer sets in which an action could be played. For this, we use a constraint:
any solution that satisfies the body of this constraint will be removed from the answer set (because the atom $\bot$ cannot be true in an answer set, as explained in Page \pageref{constraint}).
Regarding our problem, a state is eliminated if there exists a playable action in the considered state (\refll{constraintFix}{constraintFix2}). The first constraint (\refl{constraintFix}) treats the action with only one hitter and the second (\refl{constraintFix2}) the actions with $2$ hitters. For this, we developed a program that takes into account the maximum number of hitters of any action in the model and generates as many constraints as necessary.
\begin{lstlisting}
:- action(A,I,B,J,_), selectProc(A,I), selectProc(B,J), A!=B. %\label{constraintFix}
:- action(A1,I1,A2,I2,B,J,_), selectProc(A1,I1),selectProc(A2,I2), 
      selectProc(B,J), A!=B. %\label{constraintFix2}
\end{lstlisting}
Finally, the fixed points of the considered model are exactly the states represented in each remaining answer,
described by the atoms \texttt{selectProc(A,I)}.

\begin{example}[Fixed points enumeration]
The PH model of \pref{fig:ph} contains $5$ sorts:
$a$, $b$, $c$ and $d$ have $2$ processes and $b$ has 3; therefore, the whole model has $2*2*2*3*3 = 72$ states (whether they can be reached or not from a given initial state).
We can check that this model contains exactly $5$ fixed points: $\PHstate{a_1, b_0, c_0, d_0, e_1}$, $\PHstate{a_1, b_2, c_0, d_0, e_1}$, $\PHstate{a_1, b_1, c_0, d_0, e_1}$, $\PHstate{a_1, b_0, c_0, d_0, e_0}$, $\PHstate{a_1, b_2, c_0, d_0, e_0}$.
Indeed, there is no playable action in these states so no execution is possible from these. In this example, no other state verifies this property.

If we execute the ASP program detailed above (\refll{selectprocesses}{constraintFix2}),
alongside with the description of the PH model given in \pref{ex:asp-ph} (\refll{ASPprocess1}{actions2}),
we obtain $8$ answer sets that match the expected result:
%\begin{changemargin}{-2cm}{-1cm}
%\scriptsize{ 
\begin{lstlisting}[numbers=none]
Answer: 1
selectProc("a",1) selectProc("b",0) selectProc("c",0)
selectProc("d",0) selectProc("e",1)
Answer: 2
selectProc("a",1) selectProc("b",2) selectProc("c",0)
selectProc("d",0) selectProc("e",1)
Answer: 3
selectProc("a",1) selectProc("b",1) selectProc("c",0)
selectProc("d",0) selectProc("e",1)
Answer: 4
selectProc("a",1) selectProc("b",0) selectProc("c",0)
selectProc("d",0) selectProc("e",0)
Answer: 5
selectProc("a",1) selectProc("b",2) selectProc("c",0)
selectProc("d",0) selectProc("e",0)
\end{lstlisting}
%}
%\end{changemargin}
\end{example}