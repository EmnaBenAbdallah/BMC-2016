In this section, we show the effectiveness of our approach on some examples,
and compare it to other existing approaches.
All computations were performed on a Pentium~V, 3.2~GHz with 4~GB RAM.

\subsection{Evaluation}
To assess the efficiency of our new approach,
we position ourselves with respect to existing methods dealing with different biological models.
We have chosen the following tools, that are detailed below: 
\textsc{GINsim}\footnote{\textsc{GINsim} version 2.4 alpha: \url{http://ginsim.org/}} (Gene Interaction Network Simulation)~\cite{gonzalez2006ginsim,naldi2009logical,naldi2007decision};
\textsc{LibDDD}\footnote{\textsc{LibDDD} version 1.8: \url{http://move.lip6.fr/software/DDD/}}
(Library of Data Decision Diagrams)~\cite{thierry2009hierarchical,colange2013towards};
\textsc{Pint}\footnote{\textsc{Pint} version 2015-11-14: \url{http://loicpauleve.name/pint/}}~\cite{PMR12-MSCS};
and a method for CTL model-checking proposed by Rocca \textit{et al.} in~\cite{roccaasp}
which was also developed in ASP.
%but for states transitions networks.
Each method uses a specific kind of representation\footnote{When available, we used the converters included into \textsc{Pint} for these translations.}:
Thomas models (a particular kind of logical regulatory networks) for \textsc{GINsim}
and the method of Rocca \textit{et al.},
instantiable transition systems for \textsc{LibDDD},
%state transition networks for the method of Rocca \textit{et al.}
and Process Hitting (PH) for \textsc{Pint} as well as for our method.

For this comparative study, we focus on biological network of different sizes:
a tadpole tail resorption (TTR) model with 12 biological components~\cite{khalis2009smbionet},
an ERBB receptor-regulated G1/S transition (ERBB) model with 20 components~\cite{Samaga2009}
and a T-cell receptor (TCR) signaling network of 40 components~\cite{Klamt06}.
These models were chosen to be of different sizes:
from small (12 components) to large (40 components).
We note however that the considered PH models may contain more sorts than
the original number of biological components, due to the use of "cooperative sorts", which allow to model Boolean gates but do not necessarily
have a biological meaning.
The different model representations that are required to perform these benchmarks have been obtained by translations
from the PH
ensuring the conservation of the dynamical properties.

The specification of the models and the results of our enumeration of the fixed points
are summed up in \pref{tab:models}.
The time performance is roughly the same than the SAT implementation
that comes with \textsc{Pint}.
The results for several methods regarding reachability properties
are summarized in \pref{tab:reachability}.
The methods and the results provided by each of them are detailed in the following.
The overall results show that our method is efficient in computing reachability
from a given initial state;
furthermore, it sometimes provides more information than the other existing ones.

%\begin{table*}[ht]
%\begin{center}
%\noindent%
%\begin{tabular}{|l|c||c|c|c||>{\columncolor{verylightgray}}c|>{\columncolor{verylightgray}}c|}
%\hline
%  \multicolumn{2}{|c||}{Models} & \multicolumn{3}{c||}{PH representation} & \multicolumn{2}{c|}{Fixed points enumeration} \\
%\hline
%  Name & Components & Sorts & Processes & States & computation time & Nbr of results \\
%\hline
%\hline
%  TTR  & 8  & 12 & 42  & $2^{19}$ & 0.004s & 0 \\
%\hline
%  ERBB & 20 & 42 & 152 & $2^{70}$ & 0.017s & 3 \\
%\hline
%  TCR  & 40 & 54 & 156 & $2^{73}$ & 0.021s & 1 \\
%\hline
%\end{tabular}
%\vspace*{4pt}
%\caption{\label{tab:models}%
%Description of the models used in our tests and results of our fixed point enumeration.
%Each model is referred to by its short name, where
%TTR stands for the tadpole tail resorption model~\cite{khalis2009smbionet},
%ERBB for the receptor-regulated G1/S transition of the same name~\cite{Samaga2009}
%and TCR for the T-cell receptor signaling network~\cite{Klamt06}.
%For each of them, this table gives the number of biological components
%in the original representation,
%and the number of sorts, the number of processes
%and the number of states in the PH model.
%Finally, the last column gives the computation time for the enumeration of all fixed points
%and the number of results returned.
%}
%\end{center}
%\end{table*}

%\begin{table*}[ht]
%\begin{center}
%\noindent%
%\begin{tabular}{|l|l|c||c|c|c|>{\columncolor{verylightgray}}c|}
%\hline
%  \multicolumn{3}{|c||}{Experiments} & \multicolumn{4}{c|}{Results} \\
%\hline
%  & Model & Target  & \textsc{Pint} & \textsc{LibDDD} & \textsc{GINsim} & \textsc{ASP-PH} \\
%\hline
%\hline
%  \#1 & TTR & full state & 0m0.97s & 0m1.15s &  0m2.05s & 0m1.90s \\
%\hline
%  \#2 & ERBB & full state & out &1m55.38s & 2m31.64s & 0m11.84s \\
%\hline
%  \#3 & ERBB & partial state  & 0m0.03s &1m54.96s & -- & 0m5.02s \\
%\hline
%  \#4 & TCR & full state & Inconc & out & out & 6m27.93s \\
%\hline
%  \#5 & TCR & partial state & 0m0.02s & out & -- & 1m35.08s \\
%\hline
%\end{tabular}
%\vspace*{4pt}
%\caption{\label{tab:reachability}
%Compared performances of several methods to compute reachability analyses:
%The method of \textsc{Pint}, \textsc{LibDDD}, \textsc{GINsim} and our new method presented in this paper, called \textsc{ASP-PH}.
%For each test, this table gives the short name of the considered model,
%as given in table~\ref{tab:models},
%the type of goal (either a whole state or a sub-state)
%and the computation time of the different methods used for the tests,
%where ``out'' marks an execution taking too much time or memory,
%``\mbox{--}'' indicates that is not possible to do the test,
%and ``Inconc'' states that the method terminates without a response.
%}
%\end{center}
%\end{table*}

% Tableau qualitatif
% Bound
% Loop detection
% Search:

\begin{itemize}[leftmargin=*]

\item \textbf{\textsc{GINsim}} is a software for the edition, simulation and analysis
of gene interaction networks.
It allows to compute all reachable fixed points from a given initial state instantly;
however, it is not possible to compute all fixed points independently from the initial state.
Regarding the reachability problem, \textsc{GINsim} only allows to check the reachability of
full states, because its approach consists in computing
(part of) the state-transition graph and then searching for a path between the two given states.
Therefore, it was not possible to perform reachability checks on partial states
(experiments \#3 \& \#5).
Small state-transition graphs can also be displayed by this tool.

\item \textbf{\textsc{LibDDD}}
is a library for symbolic model-checking of CTL \& LTL properties.
It can thus especially be used to check reachability properties;
however, as opposed to our method, it does not output an execution path
solving this reachability.
In addition, it relies on the construction of the state-transition graph
which is then stored under the form of a binary decision diagram for a more efficient analysis.
This computation explains why \textsc{LibDDD} takes more time to respond,
and gets out of memory in about 12 minutes for the biggest example
which contains $2^{73}$ states
(experiments \#4 \& \#5).
Finally, \textsc{LibDDD} is not able to compute the fixed points of a network.

\item \textbf{\textsc{Pint}}
is a library gathering tools and converters related to the PH.
It should be noted that \textsc{Pint} contains the only reachability analysis
developed so far natively for the Process Hitting,
before the method proposed in this paper.
It consists in an approximation that avoids to compute the state-transition graph;
it is thus ensured to be really efficient, which explains the fastest results,
but at the cost of possibly terminating without being conclusive.
However, it is not designed for goals consisting of many processes,
which are more likely to trigger an inconclusive response
(such as for experiment \#4),
or an exponential research in sub-solutions
(such as for experiment \#2).
This explains the high computation times for some of the tests.
Moreover, in the case of a positive answer,
it currently does not return the execution of the path achieving the desired reachability,
but only outputs its conclusion (True, False or Inconclusive).

\item \textbf{\textsc{ASP-Thomas}}\footnote{The authors wish to thank Laurent Trilling for his patience and his help in performing the tests.}
offers the possibility to model-check CTL properties on Thomas networks.
Currently, the translation of Thomas networks into ASP, including discrete parameters, is hand-made.
Then, for a given CTL formula also described in ASP, this tool returns the set of paths satisfying the formula.
In the following, we only focus on reachability, that is, formulas of the form: $s_0 \Rightarrow \mathsf{EF}(g)$
where $s_0$ is the initial state and $g$ a goal property.
It has to be noted that this method requires to provide a total number $n$ of steps
for which the dynamics will be computed.
Indeed, all dynamics of the system are computed up to $n$ steps in order to perform a depth first verification;
thus, this value $n$ has to be bigger than the size of the minimal path and lower than the size of the maximal path,
and thus may be difficult to predict.
This is not the case for the iterative version of our method which can stop as soon as the goal is met,
and thus potentially before such a $n$\textsuperscript{th} step.
%, the program will stop at the first step that reach the goals: breath first search.
%It is clear from \pref{tab:reachability} that
%With a well-chosen number of steps, \textsc{ASP-Thomas} terminates very quickly when compared with others.
The main consequence on the \textsc{ASP-Thomas} tool
is that if one can a priori know the length of the path to reach the specified goal,
this approach can provide very good performance.
This shows that ASP can be a good choice to run the dynamics of a model and check reachability properties.
However, if the bound is totally unknown, choosing a too small path length can naturally lead to miss the goal,
and a too large one will greatly impact the performance or lead to miss the goal as well (if no path of length $n$ exist).
For example, in experiment \#3, the goal is reachable within 18 steps:
using a bound of 21, \textsc{ASP-Thomas} finishes in 2.61s;
however, with a bound of 30 steps, it takes several minutes.
We can observe similar results on the others benchmarks.
In a nutshell, with our approach, if the goal is reachable, the run time only depends on the distance to reach the goal,
while with \textsc{ASP-Thomas}, the run time only depends of the chosen bound,
because all paths of the chosen length will be generated before being checked.
A summary of differences between our approach and \textsc{ASP-Thomas} is detailed in \pref{tab:qualitative_differences}.
% This shows that ASP is a good choice to run the dynamics of a model and check reachability properties.
% The a summary of differences between our approach and \textsc{ASP-Thomas} is detailed in \pref{tab:qualitative_differences}.
% The main difference is that, in \textsc{ASP-Thomas}, the search is bounded.
% % If the number of steps to reach the goal is part of the goal or if one can a priori know the length of the path to reach the specified goal,
% % this approach can provide very good performance.
% But if the bound is totally unknown, choosing a too small path length can naturally lead to miss the goal.
% %and thus to conclude it is unreachable when it is reachable.
% On the other hand, choosing a bound that is too big will greatly impact the performance.
% For example, in experiment \#3, the goal is reachable with a path of 18 steps,
% and using a bound of 21, \textsc{ASP-Thomas} finishes in 2.61s.
% However, if we fix the bound to 30 steps, it takes several minutes.
% We can observe similar results on the others benchmarks.
% With our approach, if the goal is reachable, the run time only depends on the distance to reach the goal.
% In \textsc{ASP-Thomas}, the run time only depends of the chosen bound, because all paths of the chosen length will be generated before being checked.
\end{itemize}

%\begin{table}[ht]
%\begin{center}
%\noindent%
%\begin{tabular}{|l|c|>{\columncolor{verylightgray}}c|}
%\hline
%  & \textsc{ASP-Thomas} & \textsc{ASP-PH} \\
%\hline
%\hline
% Input Model & Thomas network & Process hitting \\
%% \hline
%%  Input generation & Hand-made & {\bf Automatic} \\ %with PH-to-ASP \\
%\hline
% Input property & CTL formula & $\mathsf{EF}$ only \\
% %$s_0 \rightarrow \mathsf{EF}(g)$ \\
%\hline
% Output & All paths & Minimal paths \\
%\hline
% Reachability & Bounded & {\bf Unbounded} \\
%\hline
% Non-reachability & Bounded & Bounded \\
%\hline
% Search & Depth first & Breadth first \\
%\hline
% Run time (small $n$) & {\bf Seconds} & Seconds/minutes \\
%\hline
% Run time (big $n$) & Out/UNSAT & {\bf Seconds/minutes} \\
%\hline
%\end{tabular}
%\vspace*{4pt}
%\caption{\label{tab:qualitative_differences}
%Qualitative comparison between the iterative version of our approach and \textsc{ASP-Thomas}.
%}
%\end{center}
%\end{table}



\subsection{Discussion}
\label{sec:discussion}

In the previous sections,
we developed two new methods in ASP to check dynamical properties,
namely identifying fixed points and finding all the shortest paths to reach a given goal.
Compared to some other methods described above
(\textsc{GINsim} and \textsc{LibDDD}) our method is relatively faster and also permits to study larger networks
(up to $2^{73}$ states in our tests). We exclusively studied networks modeled in Process Hitting. This new formalism for network modeling is a restriction of synchronous automata and thus allows to represent any kind of dynamical model. Moreover, the dynamics of PH it is easy to implement into ASP.

In practice, we can detect the loops in the model dynamics and avoid to check the same path again, as shown in \pref{sec:loops}.
However, the iterative version of \textsc{Clingo} can iterate forever even if no answer is satisfied and there is no more path to explore.
Thus, although it is expected to return an \texttt{unsatisfiable} response, the solver remains stuck for infinity in a state instead.
To our knowledge, there is no possibility yet to force the incrementation to stop in the iterative version without reaching the goals.
It is still possible, however, to limit the number of iterations to an arbitrary
maximum which will be eventually reached in such case.
This is possible with the use of a construct limiting the maximum number of steps: ``\texttt{\#const imax = n.}'',
where \texttt{n} is the maximum number of steps.

Several values can be given to this parameter \texttt{n}.
For example, the total number of states is an obvious maximum,
as it will never be exceeded by a minimum path,
but it is too hight to be very interesting.
The total number of sorts is a more interesting value,
under the hypothesis that each one will change its active process at most once,
which is often the case for Boolean networks;
or, with a similar reasoning, the total number of processes can be chosen.

Given our implementation, if the step \texttt{n} is reached (meaning no valid path was found),
the computation stops with an \texttt{unsatisfiable} response for the reachability.
But, if the goal is reached at some intermediary step, it is not necessary to continue the iteration up to ``\texttt{n}''.
Instead, the computation is stopped and the current trace of the reachability is returned.
In comparison, \textsc{ASP-Thomas} has to compute all possible dynamic evolutions of length ``\texttt{n}'' before verifying the property.
