\Maxime{Début à revoir.}
This paper presents a new logical approach to compute attractors
in a particular class of automata networks.
This class encompasses qualitative asynchronous biological networks,
making all results especially applicable to the widespread Thomas modeling.

The point of the method developed in this article is the computation and
the enumeration of the attractors of such a network.
We present a particular method for attractors of size 1 (also called fixed points or steady states)
and a more general one for all attractors of given size $N \geq 2$.
The originality of our work consists in using ASP, a powerful declarative programming paradigm.
Thanks to the encoding we introduced, and the powerful heuristics developed in modern solvers, we are able to tackle the enumeration of fixed points, cycles and attractors of medium-sized models.
The major benefit of such a method is to get an exhaustive enumeration of all potential states while still being tractable for models with dozens of interacting components.
Identifying attractors in large biological networks is a big challenge because it gives an insight to the long-term behavior of biological systems,
and we hope our work helps open new ways and tools to explore this field.

\Maxime{Il faudrait peut-être étoffer un peu la conclusion.}

\Maxime{Et ajouter des perspectives.}

%Perspective: The attractor inverse problem, which involves construct- ing BNs possessing a given attractor structure, is important to network inference from steady-state data. Most microarray- based gene-expression studies do not involve controlled tem- poral experimental data; rather, it is assumed that the data result from sampling from the steady state. Under the BN model, this means that the data come from the attractors.

%In this paper, we proposed a new logical approach to address some dynamical properties of Process Hitting models.
%The originality of our work consists in using ASP, a powerful declarative programming paradigm.
%Thanks to the encoding we introduced, we are not only able to tackle the enumeration of fixed points but also to check reachability properties.
%The major benefit of such a method is to get an exhaustive enumeration of all corresponding paths while still being tractable for models with dozens of interacting components.
%
%One of the perspectives of our work is to extend the set of models on which our approach could be applied. We can consider the addition of priorities or neutralizing edges, or tackle other representations,
%such as Thomas modeling~\cite{Thomas73}.
%However, the range of the analysis can also be enriched,
%by searching the set of initial states
%allowing to reach a given goal instead of the other way around,
%or extending the method to universal properties (like the $\mathsf{AF}$ operator in CTL).
