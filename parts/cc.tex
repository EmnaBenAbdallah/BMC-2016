This paper a new logical approach to compute attractors in automata networks modeled in Process Hitting. We have developed a new idea to identify the attractors whatever their sizes. The originality of our work consists in using ASP, a powerful declarative programming paradigm. Thanks to the encoding we introduced, we are able to tackle the enumeration of fixed points, cycles and attractors. The major benefit of such a method is to get an exhaustive enumeration of all searching states while still being tractable for models with dozens of interacting components. It is a big challenge that how to identify attractors in the large biological networks. Based on above research, it was proved that we can resolve this problem thanks to the modeling way of biological networks (PH) and the logical programming (ASP).


%In this paper, we proposed a new logical approach to address some dynamical properties of Process Hitting models.
%The originality of our work consists in using ASP, a powerful declarative programming paradigm.
%Thanks to the encoding we introduced, we are not only able to tackle the enumeration of fixed points but also to check reachability properties.
%The major benefit of such a method is to get an exhaustive enumeration of all corresponding paths while still being tractable for models with dozens of interacting components.
%
%One of the perspectives of our work is to extend the set of models on which our approach could be applied. We can consider the addition of priorities or neutralizing edges, or tackle other representations,
%such as Thomas modeling~\cite{Thomas73}.
%However, the range of the analysis can also be enriched,
%by searching the set of initial states
%allowing to reach a given goal instead of the other way around,
%or extending the method to universal properties (like the $\mathsf{AF}$ operator in CTL).
