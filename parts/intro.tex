\Emna{A completer avec les bonne références}
\Maxime{Ajouter quelques généralités sur les BRN}

\Maxime{Je ne suis pas d'accord sur la notion de “cyclic attractor”: je crois que tu confonds “trap domain” (un ensemble d'états dont on ne peut pas s'échapper ; un attracteur est un trap domain minimal) et “cyclic attractor” (qui est un attracteur, donc minimal, composé d'un seul cycle).}

In recent decades, rapidly evolving new technologies have been producing a massive amount of biological data (genomics, proteomics...). This lead to dramatic developments in systems biology which could sustain on this data. Indeed, from the perspective of understanding the nature of a cellular function, it is essential to study the behavior of its components (genes, DNAs, proteins...) with a global view rather than an individual one, and take into account the fact that activities and expressions of cellular components are not isolated but dependent on each other.

In this focus, the long-term behavior of regulatory networks dynamics is of particular interest~\cite{wuensche1998genomic}.
Indeed, at any moment, a system lies into a certain \emph{basin of attraction}, which is an area of its dynamics that cannot be escaped.
When evolving, the system eventually falls into a new and smaller basin of attraction, thus reducing its of possible future behaviors as previous ones become unreachable.
This phenomenon may depend on biological switches or other complex phenomena
and it explains why, for any initial condition and on the long run, any system will eventually evolve into a final situation in which the number of possible behaviors is the most restricted. \Emna{ref?}
The system has then reached a minimal basin of attraction.
Such constructs have been interpreted as distinct responses of the organism, such as distinct cell types in multicellular organisms. \Maxime{Ref ? Peut-être détailler un peu plus pour insister sur l'intérêt des attracteurs ?}

Various kinds of mathematical models have been proposed for the modeling of Biological Regulatory Networks (BRNs). These models include neural networks, differential equations, Petri nets, Boolean networks (as proposed by Kauffman \Emna{ref}), probabilistic Boolean networks, and other multi-valued models such Synchronous Automata Networks. In this paper, we focus on Process Hitting \Maxime{ref}, an asynchronous formalism which is a restriction of Automata Networks, and that especially encompasses the framework of René Thomas \Emna{ref}. Indeed, qualitative frameworks have received substantial attention, because of their capacity to capture the switching behavior of genetic or biological processes, and therefore their long-term behavior. This is why we use a qualitative representation for the study of basins of attraction.
More particularly, in a qualitative framework, a minimal basin of attraction can take two different forms: it can be either a \emph{stable state} (a state from which the system doesn't evolve anymore, also called a fixed point) or an \emph{attractor} (a minimal set of states that loops indefinitely and cannot be escaped).

%%%%%%%%%

The computational problem of finding all attractors in a BRN is difficult. Even the simpler problem of deciding whether the system has a fixed point (which can be seen as the smallest possible kind of attractor) is NP-hard.
Based on the former \Maxime{Il manque une ref ?}, many studies have proven that the computational way of attractors in BRNs is a NP-hard problem. \Maxime{ref ?}

Therefore, it is meaningful to identify attractors. Of course, these can be done by examining all possible states of a network with an exhaustive approach. For instance, finding an attractor or a stable state can be performed by randomly selecting an initial state and following a long enough trajectory.
However, listing all attractors with this method can be too time consuming even for small networks: $2^n$ initial states have to be considered for a Boolean networks of size $n$; and multi-valued networks raise this complexity even more. Furthermore, enough computation have to be performed to ensure that all trajectories have been explored and all attractors were found.
%%%%%%%

\Emna{Les travaux qui ont été faits concernant la recherche des attracteurs}

Our goal in this paper is to develop exhaustive methods to analyze a Biological Regulatory Network (BRN) modeled in Process Hitting (PH). With respect to PH dynamics, we exhibit three kinds of results:
\begin{itemize}
\item[-] Finding all possible steady states of a BRN,
\item[-] Computing all cycles with fixed size $N \in \mathbb{N}$,
\item[-] Enumerating all attractors of size $N \in \mathbb{N}$.
\end{itemize}
The particularity of our contribution relies in the use of Answer Set Programming
(ASP) \cite{baral2003knowledge}
to compute the results.
This declarative programming framework has proved efficient
to tackle models with a large number of components and parameters.
Our aim here is to assess its potential w.r.t.\ the computation
of some dynamical properties of PH models.
We chose the PH framework because it allows to represent a wide range of dynamical multi-valued models in the asynchronous framework, and the particular form of its actions
can be easily represented using ASP,
with exactly one fact per action.

This paper is organized as follows.
\pref{sec:defs} presents the main definitions related to the Process Hitting and the particular constructs that we will seek: fixed points and attractors,
and \pref{sec:asp} briefly present the Answer Set Programming.
\pref{sec:fixpoint} details our method that allows to translate a PG into ASP and find all the fixed points in such a model
and \pref{sec:attractors} explains how to enumerate all cycles in a model and use this result to find all the attractors.
\pref{sec:results} sums up the results obtained with our methods on several models of different sizes.
Finally, \pref{sec:ccl} concludes and gives some perspectives to this work.
