\subsection{Intuitive idea}
The computational method for finding attractors of size $N$ in an automata network falls into three stages:
\begin{enumerate}
  \item Enumerating all sub-State Transition Graphs (sub-STGs)
    with $N$ transitions,
  \item Filtering out all of these sub-STG that are not cyclic,
  \item Checking, for each state of the remaining cycles
    if all the next states are part of this cycle.
\end{enumerate}
Once all steps are passed, the only remaining sub-STGs are attractors of size $N$,
because they consist of cycles that cannot be escaped.

% Description de l'algo
\Maxime{Idem : expliquer les raisons de mettre l'algo.}
The pseudocode of the presented algorithm is given as Algorithm \ref{alg:PH-attractor}. We use ASP for the identification of all sub-STG with $N$ transitions in a multiple-valued network. \Maxime{C'est quoi un “multiple-valued network” ? “Multi-valued” ? Mais pourquoi le préciser ici, pourquoi ne pas juste dire “PH” ?} First, we generate a set of loops of size $N$. Then, a set satisfying the attractors formula corresponds to a valid solution. This process can be repeated iteratively for larger and larger values of $N$ until all attractors are identified.

%Faire un algo
\begin{algorithm}[!h]
	\caption{Enumarate Attractors of size $N$ from an Automata Network}
	\label{alg:PH-attractor}
	\begin{itemize}
		\item[] \textbf{INPUT}: Automata Network $PH=(\PHs,\PHl,\PHa)$, Attractor size $N$ // $N \geq 2$
		\item[] \textbf{OUTPUT}: All the attractors of size $N$
		\item[] \textbf{Initialize}: $\Delta_N \longleftarrow \varnothing$ // The set of the attractors of size $N$
		\item[] \textbf{Begin} \\
		
			\hspace{0.2cm}	\textbf{\textit{forall}} $\PHst_i \in \PHl$ \textbf{\textit{do}} 
				
				\begin{itemize}
					\item[] compute the $N$ successors: $\PHst_i \rightarrowtail ... \rightarrowtail \PHst_N$ \\ let $\Delta_N^i=$\{$\PHst_j \in \PHl, i \leq j \leq N$\} %be the set of these states
					
					\item[] \textbf{\textit{if}} $\PHst_i = \PHst_N$ \textbf{\textit{then}} \\
						\hspace{0.7cm}  $loop(\Delta_N^i) \longleftarrow True$ \\
					 \textbf{\textit{else}} \\
					 	\hspace{0.7cm}  $loop(\Delta_N^i) \longleftarrow False$ \\
					 \textbf{\textit{end if}} 
					 
					\item[] \textbf{\textit{if}} $loop(\Delta_N^i) = True$ \textbf{\textit{then}} \\
					
%						\begin{itemize}
							\hspace{0.7cm} $attractor(\Delta_N^i) \longleftarrow True$	\\
							
							\hspace{0.7cm} \textbf{\textit{forall}} $\PHst_j \in \Delta_N^i$ \textbf{\textit{do}} \\
							
							\hspace{1.2cm} \textbf{\textit{if}} $\exists$ $\Sce(\PHst_j) \notin  \Delta_N^i$ \textbf{\textit{then}} \\ \hspace{1.7cm} $attractor(\Delta_N^i) \longleftarrow False$ \\
							\hspace{1.2cm} \textbf{\textit{end if}} 
							
							\textbf{\textit{else}} \\
							\hspace{0.7cm} $attractor(\Delta_N^i) \longleftarrow False$ \\
%						\end{itemize}		
							\textbf{\textit{end if}} 
														
					\item[] \textbf{\textit{if}} $attractor(\Delta_N^i) = True$ \textbf{\textit{then}}\\
						\hspace{0.7cm} \textbf{\textit{forall}} $\PHst_j, \PHst_k \in \Delta_N^i, j \neq k$ \textbf{\textit{do}} \\								
								\hspace{1.2cm} \textbf{\textit{if}} $loop(\Delta_k^j) = True$ \textbf{\textit{then}}\\
									\hspace{1.7cm} $complexLoop(\Delta_N^i) \longleftarrow False$ \\
									\hspace{1.7cm} $x \longleftarrow j+1$ \\									
									\hspace{1.7cm} \textbf{\textit{while}} $x \leq k-1 $ \textbf{\textit{and}} $complexLoop(\Delta_N^i) = False$ \textbf{\textit{do}} \\	
									
										\hspace{2.1cm} \textbf{\textit{if}} $loop(\Delta_{k+x}^{j+x}) = False$ \textbf{\textit{then}}\\
											\hspace{2.5cm} $complexLoop(\Delta_N^i) \longleftarrow True$ \\
										\hspace{2.1cm} \textbf{\textit{end if}} \\
									\hspace{1.7cm} \textbf{\textit{end while}} \\
								\hspace{1.2cm} \textbf{\textit{end if}} \\
						\hspace{0.7cm} \textbf{\textit{end forall}} 	\\

					\hspace{0.7cm} \textbf{\textit{if}} $complexLoop(\Delta_N^i) = False$ \textbf{\textit{then}}\\
						\hspace{1.2cm}  $attractor(\Delta_N^i) \longleftarrow False$\\
					\hspace{0.7cm} \textbf{\textit{end if}} 						
						
					\textbf{\textit{end if}} \\

					\item[] \textbf{\textit{if}} $attractor(\Delta_N^i) = True$ \textbf{\textit{then}}\\
					\hspace{0.5cm}  $\Delta_N \longleftarrow \Delta_N \cup \Delta_N^i$ \\
					\textbf{\textit{end if}} 

				\end{itemize}		
			\hspace{0.2cm} \textbf{\textit{end forall}} \\		
			\hspace{0.2cm} \textbf{\textit{return}} $\Delta_N$		
	\end{itemize}
\end{algorithm}

% Mettre du code ASP + explication
\subsection{Cycles enumeration}
The presented algorithm searches for all sub-graphs with a given number $N$ of transitions %(i.e path of length $N$)
in the network. As we start each time from a different initial state to simulate the evolution of the model, it means that in the end we browse all the model and generate all possible states (\refl{selectprocesses}). 

The evolution of a model is done in a limited number of steps $N$. At the beginning of an ASP program we can define constants whose values will be choosen by the user every time he execute it. Here we use \texttt{n} as a constant that defines the number of states we want to have.
We therefore define the predicate \texttt{state(0..n)}.
For example, \texttt{state(0..5)} will compute the 6 successive states, including the initial state.

The dynamics of a network is described by its actions; therefore, identifying the future states requires to first identify the playable actions for each state. We remind that an action is playable in a state when all its hitter processes and target process are active in this state (see Definition \ref{def:playableAction}). Therefore, we define an ASP predicate $\texttt{playable(action(A}_\texttt{1}\texttt{,I}_\texttt{1}\texttt{,...,A}_\texttt{n}\texttt{,I}_\texttt{n}\texttt{, B,J,K),S)}$ which is true when the processes  ${\texttt{A}_\texttt{1}}_{\texttt{I}_\texttt{1}}\texttt{,...,}{\texttt{A}_\texttt{n}}_{\texttt{I}_\texttt{n}}$ and $\texttt{B}_\texttt{J}$ are active at state \texttt{S}.

The cardinality rule of \refl{e2a}
creates a set of as many predicates as there are possible simulations from the chosen state,
thus reproducing all possible branchings in the possible simulations of the model, in the form of as many potential answer sets. It is also needed to enforce the strictly \emph{asynchronous} dynamic
which forces that exactly one process can change between two states and keeping the non-deterministic property.
To remove all scenarios where two or more actions have been played between
two states, we use the constraint of \refl{e2b}.
Thus, the remaining scenarios contained in the answer sets all strictly follow
the asynchronous dynamics of the Automata Networs.
We finally witness that the active expression level of sort \texttt{B} has changed between states \texttt{S} and \texttt{S+1} with the predicate \texttt{change(B,S+1)} of \refll{e3a}{e3b}. We create as much as necessary rules to find the changed sort that is the result of an action having one or more hitters (here we detailed action with 1 and 2 hitters).

\begin{lstlisting}
1{play(Action,S)}1 :- playable(Action,S), state(S). %\label{e2a} 
:- 2{play(Action,S)}, state(S). %\label{e2b}
change(B,S+1) :- play(action(_,_,B,_,_),T), state(S). %\label{e3a}
change(B,S+1) :- play(action(_,_,_,_,B,_,_),T), state(S). %\label{e3b}
\end{lstlisting}

Finally, the active processes at state \texttt{S+1},
thus representing the next state in the dynamics, depending on the chosen bounce,
can be computed by the rules of \refll{e4}{e5a}.
\begin{lstlisting}
instate(activeProc(B,K),S+1) :-  play(action(_,_,B,_,K),S), state(S).%\label{e4}
instate(activeProc(B,K),S+1) :-  instate(activeProc(B,K),S),%\label{e5}
				  not change(B,S+1), state(S).%\label{e5a}
\end{lstlisting}
In other words, the state of state \texttt{S+1} contains one updated active process $\texttt{B}_\texttt{K}$
resulting from the predicate \texttt{play(action(\_,\_,B,\_,K),T)} (\refl{e4})
as well as all the unchanged processes that correspond to the other sorts (\refll{e5}{e5a}).

The overall result of the pieces of program presented in this subsection
is an answer set containing one answer for each
possible simulation in \texttt{n} time steps,
starting from each state.
As a result, it reproduces all sub-STG of size $N$.
%%%%%%%%%%%

Once a sub-STG is found, we check whether it is a cycle (i.e a loop) or not. 
To do so, we use the predicate \texttt{getNbrRepetition(X,S1,S2)} which permits to get the number \texttt{X} of identical active processes between two tates \texttt{S1} et \texttt{S2}. This predicate refers to another predicate, \texttt{repeter(P,S1,S2)} (\refll{repeter1}{repeter2}), that is true for each process which is active at both \texttt{S1} and \texttt{S2}. 
\begin{lstlisting}
repeter(P,S1,S2) :- selectProcSt(P,S1), selectProcSt(P,S2), %\label{repeter1}
state(S1), state(S2). %\label{repeter2}
getNbrRepetition(X,S1,S2) :- X={repeter(F,S1,S2)}, S1!=S2, %\label{getRep1}
state(S1), state(S2). %\label{getRep2}
\end{lstlisting}
Then computing the number of repeated processes between \texttt{S1} and \texttt{S2} in \refll{getRep1}{getRep2} will permit to us to compare this number with the total number of sorts, which is the maximum number of repetition, and determine if there is a loop between the two states. In our case we just need to compare the first state (\texttt{S1=0}) and the last state (\texttt{S2=n}) of the sub-STG. Therefore we eliminate all sub-STG that are not cycles by the constraint (\refl{loop}).
\Maxime{Cette partie mériterait d'être un peu mieux expliquée.}
\begin{lstlisting}
:- loop(0, n)  %\label{loop}
\end{lstlisting}

\subsection{Attractors enumation}
\Maxime{Tout le début de cette section n'est pas très clair et mériterait aussi d'être clarifié et reformulé.}
Since each state of the STG of an automata network has various next states (due to the non-deterministic dynamics), a cyclic sub-STG is not necessarily an attractor. Thus, we first need to determine the presence of a cycle by checking whether the first state and the last state of the sub-STG are the same. Then we check if this sub-STG is an attractor. 
Filtering the remaining sub-STGs means using a constraint in ASP. Indeed a constraint has many advanges for such NP-Hard problems. Indeed, it permits to decrease the computational time. But to benifit of this specific type of logic rules, we have to think oppositly and present the opposite of what we are looking for. 
In our case, to say that a cycle is not an attractor, it means that it has at least one state that has a successor which is not apart of it (\refl{existOut}).
\begin{lstlisting}
:- existActionOut.  %\label{existOut}
\end{lstlisting}
It is necessary now to compute all playble actions for each state $\texttt{S}_\texttt{i}$ between the states \texttt{0} and \texttt{n}. For that we propose a similar rule to \refl{e2a} but with different predicate name and different parameters: \texttt{candidatePlayable(Action,S)}. It is true when an action \texttt{Action} is playable at state \texttt{S} which is between \texttt{0} and \texttt{n}. Also this action should be different from the action that was played in the cycle transforming the system from \texttt{S} to \texttt{S+1} that is also a part of the cycle (\refll{comptPlayAct}{candidatPlayAct}).
\begin{lstlisting}
firstCandidatePlayable(action(A,I,B,J,K),S) :- action(A,I,B,J,K), %\label{comptPlayAct}
		selectProcSt(activeProc(B,J),S), selectProcSt(activeProc(A,I),S), 
		loop(0, n), S>0, S<n, state(S).		
candidatePlayable(Action1,S) :- firstCandidatePlayable(Action1,S), 
		playable(Action2,S), Action1!= Action2.	%\label{candidatPlayAct}
\end{lstlisting}
After finding the candidate playable actions in each state of the cycle, it remains to verify if the result states are already a part of the cycle or if there are new states. Indeed it is possible to have other playable actions in the cycle (\texttt{playableInCycle}, \refll{playIncyc1}{playIncyc3}) but these actions lead to the same states. It confirms that the model indefinitely cycles in these states. Once all playable actions are checked whatever they leads to a state in the cycle or not, the predicate \texttt{existActionOut} is verified when at least one action could envolve the system to a state out of the cycle (\refl{existActOut1}).
\begin{lstlisting}
playableInCycle(action(A,I,B,J,K),S) :- candidatePlayable(action(A,I,B,J,K),S), %\label{playIncyc1}
	selectProcSt(activeProc(B,K),Si), Si>=0, Si<=n, Si!=S+1, %\label{playIncyc2}
	getNbrRepetition(X,S,Si), getNbreSorts(Y), Y==X+1, state(Si). %\label{playIncyc3}
existActionOut :- candidatePlayable(Action,S), not playableInCycle(Action,S). %\label{existActOut1}
\end{lstlisting}

At this step the question is how to be sure if this attractor of size $N$ is not a set of attractors of size lowest than $N$. The idea is (like descriped in the algorithm \ref{alg:PH-attractor} and lines \ref{loopX}-\ref{constraintloopX}) to do not consider the repeated loops as an attractor. We compute in the predicate \texttt{loopX(S1, S2, X)} if there is as well a loop between the $X^{th}$ successor of \texttt{S1} and the $X^{th}$ successor of \texttt{S2}. It means propably  that a same loop is repeated. In \texttt{notloopX(S1, S2)} we compute if there is at least one $X^{th}$ successor of \texttt{S1} does not have a loop with the $X^{th}$ successor of \texttt{S2}. Finally with the constraint in line \ref{constraintloopX}, we eliminate the case of having the repeated loops.

\begin{lstlisting}
loopX(S1,S2,X) :- loop(S1+X, S2+X), 0<=X, X<=S2-S1, state(S1;S2;X). %\label{loopX}
notloopX(S1,S2) :- not loopX(S1,S2,X), 0<=X, X<=S2-S1, state(S1;S2;X). %\label{notloopX}
 	:- loop(S1,S2), loop(S1, S2+(S2-S1)), not notloopX(S1,S2). %\label{constraintloopX}
\end{lstlisting}

The problem of finding attractors in a multi-valued automata network is NP-hard. However, we have good results when the algorithm \ref{alg:PH-attractor} is developed with ASP \Maxime{Je ne suis pas d'accord : on ne traduit pas cet algorithme en ASP ! On écrit autre chose qui doit nous donner un résultat équivalent, mais ce n'est pas un algorithme impératif}. Indeed, we show in Section \ref{sec:results}, by performing a number of computational experiments on biological networks, that the program can tackle even large systems and return a result in few seconds.

%
%
%Clearly, all states between any two occurrences of the last state belong to a loop. A loop corresponds to an attractor. We mark all attractor’s states. In the following iterations, we will only search for paths in which the last state is not marked.
%Until at least one attractor remains unmarked, we can find a path of any length since we can cycle in an attractor forever. However, once all attractors are identified and marked, we will only be able find paths which are shorter than a given length (at most the diameter of the STG). So, when we search for a path of some length k and it does not exist, this means that all attractors are already identified and the algorithm can terminate.
%If a path of length k does exist and it is loop-free, we double k and continue the search for a path of the new length.
%We illustrate the algorithm on the example of the Boolean network in Figure 1. The algorithm starts from searching for a path of length $N = 3$
