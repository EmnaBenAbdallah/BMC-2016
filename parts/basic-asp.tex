\Emna{A revoir et parler d'avantage des contraintes}
In this section, we briefly recapitulate the basic elements of Answer Set Programming (ASP) \cite{baral2003knowledge}, a  declarative language that proved efficient to address highly computational problems. An answer set program is a finite set of rules of the form:
\begin{equation}
\label{asprule}
  a_{0}\ \leftarrow \ a_{1},\ \ldots,\ a_{m},\ not\ a_{m+1},\ \ldots,\ not\ a_{n}
\end{equation}
where $n \ge m \ge 0$, $a_{0}$ is a propositional atom or $\bot$, all
$a_{1}, \ldots ,a_{n}$ are propositional atoms, and the symbol ``$not$'' denotes  negation as failure.
The intuitive reading of such a rule is that whenever $a_{1}, \ldots, a_{m}$
are known to be true and there is no evidence for any of the negated atoms $a_{m+1}, \ldots, a_{n}$ to be true, then $a_{0}$ has to be true as well.
\label{constraint}
If $a_{0} = \bot$, then the rule becomes a constraint (in which case the symbol $\bot$ is usually omitted).
As $\bot$ can never become true, if the right-hand side of a constraint is validated, it invalidates the whole answer set.

In the ASP paradigm, the search of solutions consists in computing the \emph{answer sets} of a given program.
An answer set for a program is defined by Gelfond and Lifschitz \cite{DBLP:conf/iclp/GelfondL88} as follows.
An \emph{interpretation} $I$ is a finite set of propositional atoms.
%An atom $a_i$ is \emph{true under $I$} if $a_i \in I$, and false otherwise.
A rule $r$ as given in \eqref{asprule} is \emph{true under~$I$} if and only if:
  \[\{a_1,\ \dots,\ a_{m}\} \subseteq I \wedge \{a_{m+1},\ \ldots,\ a_{n}\} \cap I = \emptyset \Rightarrow a_{0} \in I \enspace.\]
An interpretation $I$ is a \emph{model} of a program $P$ if each rule $r \in P$ is true under $I$.
Finally, $I$ is an answer set of $P$ if $I$ is a minimal (in terms of inclusion) model of $P^{I}$,
where $P^{I}$ is defined as the program that results from $P$ by deleting all rules that contain a negated atom that appears in $I$,
and deleting all negated atoms from the remaining rules.
Programs can yield no answer set, one answer set, or several answer sets.
To compute the answer sets of a given program, one needs a grounder (to remove variables from the rules) and a solver.
For the present work, we used \textsc{Clingo}\footnote{We used \textsc{Clingo} version 4.5.0: \url{http://potassco.sourceforge.net/}}~\cite{gebser2010incremental} which is a combination of both.

In section \pref{sec:defs} we detailed the dynamic properties (fixed points, cycles and attractors). How to verify them with the help of ASP is the subject of the rest of this paper.
